\documentclass[12pt]{article}

\usepackage[utf8]{inputenc}
\usepackage[greek,english]{babel}
\usepackage{alphabeta}
\usepackage{latexsym, amsfonts, amssymb, amsthm, amsmath, titlesec, stackrel, mnsymbol}

\newcommand{\R}{\mathbb{R}}
\newcommand{\N}{\mathbb{N}}
\newcommand{\norm}[1]{\left\lVert#1\right\rVert}
\newcommand{\margin}{\hspace{4pt}}


\newenvironment{rcases}
    {\left.\begin{aligned}}
    {\end{aligned}\right\rbrace}

\setlength{\parindent}{0in}
\setlength{\oddsidemargin}{0in}
\setlength{\textwidth}{6.5in}
\setlength{\textheight}{10in}
\setlength{\topmargin}{-1.0in}
\setlength{\headheight}{18pt}

\titlespacing*{\subsection}
{0pt}{5.5ex plus 1ex minus .2ex}{4.3ex plus .2ex}

\title{\hugeΑλγοριθμική Επιχειρησιακή Έρευνα\\Πρώτη Εργασία}
\author{Σιώρος Βασίλειος\\Ανδρινοπούλου Χριστίνα}
\date{Οκτώβριος 2019}

\begin{document}

\maketitle
\thispagestyle{empty}

\pagebreak

\pagenumbering{arabic}

\subsection*{1. Let $C \subseteq \R^n$ be a \textit{convex} set with $x_1 , \dotsc, x_k \in C$ and let $θ_1 , . . . , θ_k \in \R$ satisfy $θ_i \geq 0$ and
$θ_1 + \dotsb + θ_k = 1$. Show that $θ_1x_1 + \dotsb + θ_kx_k \in C$.}

Θα αποδείξουμε την παραπάνω πρόταση με επαγωγή στο $k$.\\

Για $k = 2$ ισχύει από τον ορισμό του \textit{convex} συνόλου,
καθώς κάθε \textit{convex} σύνολο περιέχει το ευθύγραμμο τμήμα, που ορίζεται από δύο οποιαδήποτε σημεία του.\\

ΒΑΣΗ ΕΠΑΓΩΓΗΣ: Για $k = 3$ έχουμε 

\begin{align*}
    θ_1 \cdot x_1 + θ_2 \cdot x_2 + θ_3 \cdot x_3 & = x \\ 
    θ_1, θ_2, θ_3 & \geq 0 \\
    θ_1 + θ_2 + θ_3 & = 1
\end{align*}

Σίγουρα ένα εκ των $θ_1, θ_2, θ_3$ θα είναι διάφορο του 1. Χωρίς βλάβη της γενικότητας υποθέτουμε ότι 
$θ_1 \neq 1$, επομένως

\begin{align*}
    x = θ_1 \cdot x_1 +  θ_2 \cdot x_2 + θ_3 \cdot x_3 & = θ_1 \cdot x_1 + (1 - θ_1) \cdot λ_2 \cdot x_2 + (1 - θ_1) \cdot λ_3 \cdot x_3 \\
    λ_2 & = \frac{θ_2}{(1 - θ_1)} \\
    λ_3 & = \frac{θ_3}{(1 - θ_1)}
\end{align*}

Αν το ευθύγραμμο τμήμα, που ορίζεται από τα $x_2$ και $x_3$ ανήκει στο C, τότε και το

\begin{align*}
    θ_1 \cdot x_1 + (1 - θ_1) \cdot (λ_2 \cdot x_2 + λ_3 \cdot x_3)
\end{align*}

θα ανήκει στο C. Θέλουμε να δείξουμε ότι $y = λ_2 \cdot x_2 + λ_3 \cdot x_3$ ανήκει στο C.

\begin{align*}
    λ_2 + λ_3 & = \frac{θ_2}{(1 - θ_1)} + \frac{θ_3}{(1 - θ_1)} = \frac{θ_2 + θ_3}{(1 - θ_1)} = \frac{(1 - θ_1)}{(1 - θ_1)} = 1
\end{align*}

Συνεπώς, αφού $x_2, x_3 \in C$, το $y$ ανήκει στο C, άρα και το x ανήκει στο $C$, αφού $x_1 \in C$. \\

ΕΠΑΓΩΓΙΚΗ ΥΠΟΘΕΣΗ: Υποθέτω ότι $θ_1 \cdot x_1 + \dotsb + θ_{n-1} \cdot x_{n-1}$ ανήκει στο C με $θ_i \geq 0$ για $1 \leq i \leq n-1$ και $ \sum_{i=1}^{n-1}θ_i = 1 $ ανήκει στο C. \\

ΕΠΑΓΩΓΙΚΟ ΒΗΜΑ: Θα δείξουμε ότι $z = θ_1 \cdot x_1 + \dotsb + θ_{n-1} \cdot x_{n-1} + θ_n \cdot x_n$ ανήκει στο $C$. \\

Χωρίς βλάβη της γενικότητας, υποθέτουμε ότι $θ_n \neq 1$\\

\begin{align*}
    z & = θ_n \cdot x_n + (1 - θ_n) \cdot (λ_1 \cdot x_1 + \dotsb + λ_{n-1} \cdot x_{n-1}) \\
    λ_i & = \frac{θ_i}{(1 - θ_n)} \margin \forall \in \lbrace 1, \dotsc, n \rbrace
\end{align*}

Όμως, $λ_1 \cdot x_1 + \dotsb + λ_{n-1} \cdot x_{n-1}$ από επαγωγική υπόθεση ανήκει στο C. \\

Άρα, και το z ανήκει στο $C$.

\vspace{2in}

\pagebreak

\subsection*{2. Show that a set is \textit{convex} if and only if its intersection with any line is \textit{convex}.}

$\bullet$ Αν έχουμε ένα \textit{convex} σύνολο, τότε η τομή του \textit{convex} συνόλου με μία τυχαιά ευθεία είναι \textit{convex}. \\

Αρχικά, να επισημανθεί ότι μία ευθεία είναι \textit{convex} σύνολο,
καθώς αν πάρουμε δύο οποιαδήποτε σημεία που ανήκουν στην ευθεία,
τότε ανήκει και το ευθύγραμμο τμήμα που ορίζεται από αυτά.
Συνεπώς, η ευθεία είναι \textit{convex} σύνολο. \\

Θα αποδείξουμε ότι η τομή δύο \textit{convex} συνόλων είναι \textit{convex} σύνολο \\

Έστω Α και Β είναι δύο \textit{convex} σύνολα. \\

$\forall p_i, p_j \in (A \cap B)$ το ευθύγραμμο τμήμα $\overlinesegment{p_{i}p_{j}}$ ανήκει εξ ολοκλήρου στο Α, 
επειδή Α είναι \textit{convex} σύνολο και ανήκει εξ ολοκλήρου στο Β, 
επειδή Β είναι \textit{convex} σύνολο, 
άρα ανήκει εξ ολοκλήρου και στο $Α \cap Β$. 
Συνεπώς, το $Α \cap Β$ είναι \textit{convex} σύνολο. \\

Συνεπώς, η τομή ενός \textit{convex} συνόλου και μιας ευθείας είναι \textit{covex} σύνολο. \\

$\bullet$ Aν η τομή ενός συνόλου με οποιοαδήποτε ευθεία είναι \textit{convex} σύνολο, 
τότε το σύνολο είναι \textit{convex}. \\

$\forall x_1, x_2$ που ανήκουν στο σύνολο, η τομή του συνόλου με την ευθεία, 
που ορίζεται από τα  σημεία $x_1$ και $x_2$ είναι convex, άρα \\

$\forall \theta$ με $0 \leq \theta \leq 1$ τα σημεία 
$\theta \cdot x_1 + (1 - \theta) \cdot x_2$ ανήκουν στην τομή τους 
κι ως εκ τούτου και στο σύνολο. \\

Συνεπώς, το σύνολο είναι convex.

\vspace{2in} %Leave more space for comments! \log(\frac{1}{\delta})

\pagebreak

\subsection*{3. Show that a set is \textit{affine} if and only if its intersection with any line is \textit{affine}.}

$\bullet$ Έστω ένα \textit{affine} σύνολο S. Θα δείξουμε ότι η τομή του με οποιαδήποτε ευθεία είναι \textit{affine} σύνολο.\\

Γνωρίζουμε ότι κάθε ευθεία είναι \textit{affine} σύνολο.
Θεωρούμε τυχαία ευθεία l και βαφτίζουμε $S_l$ την αναπαράστασή της ως \textit{affine} σύνολο.\\

Έστω τώρα $x_1, x_2 \in S \cap S_l$.
Δεδομένου ότι τα $S$ και $S_l$ είναι \textit{affine} σύνολα έχουμε\\

\begin{align*}
    &\begin{rcases}
        x_1, x_2 \in S \cap S_l \implies x_1 \in S \wedge x_2 \in S & \implies \theta_1 \cdot x_1 + \theta_2 \cdot x_2 \in S \margin \forall \theta_1, \theta_2 \in \R\\
        x_1, x_2 \in S \cap S_l \implies x_1 \in S_l \wedge x_2 \in S_l & \implies \theta_1 \cdot x_1 + \theta_2 \cdot x_2 \in S_l \margin \forall \theta_1, \theta_2 \in \R
    \end{rcases}
    \Rightarrow \\ \\
    &x_1, x_2 \in S \cap S_l \implies \theta_1 \cdot x_1 + \theta_2 \cdot x_2 \in S \cap S_l \margin \forall \theta_1, \theta_2 \in \R
\end{align*}

Επομένως, η τομή του \textit{affine} συνόλου $S$ με οποιαδήποτε ευθεία είναι \textit{affine} σύνολο.\\

$\bullet$ Έστω σύνολο $S$, τέτοιο ώστε η τομή του με οποιαδήποτε ευθεία να είναι \textit{affine} σύνολο.
Θα δείξουμε ότι το σύνολο $S$ είναι \textit{affine}.\\

Όπως αναφέρθηκε και προηγουμένως, κάθε ευθεία είναι \textit{affine} σύνολο.
Έστω, το \textit{affine} σύνολο $S_l$, που αντιστοιχεί στην ευθεία $l$
που ορίζεται από δύο τυχαία σημεία $x_1, x_2 \in S$.\\

Αφού, είναι γνωστό από υπόθεση πως το σύνολο $S \cap S_l$ είναι \textit{affine}, έχουμε

\begin{align*}
    x_1, x_2 \in S \cap S_l \implies  \theta_1 \cdot x_1 + \theta_2 \cdot x_2 \in S \cap S_l \margin \forall \theta_1, \theta_2 \in \R
\end{align*}

και ως εκ τούτου

\begin{align*}
    x_1, x_2 \in S \implies  \theta_1 \cdot x_1 + \theta_2 \cdot x_2 \in S \margin \forall \theta_1, \theta_2 \in \R
\end{align*}

Επομένως, το σύνολο $S$ είναι \textit{affine}.

\vspace{2in} %Leave more space for comments! \log(\frac{1}{\delta})

\pagebreak

\subsection*{4. A set C is midpoint \textit{convex}, if whenever two points $a, b \in C$, the average or midpoint
$(a + b) \text{ / } 2 \in C$. Obviously, a \textit{convex} set is midpoint \textit{convex}. Prove that if C is closed and
midpoint \textit{convex}, then C is \textit{convex}.}

Έστω σημεία $a, b \in C$ και m ένα οποιαδήποτε σημείο επί του ευθύγραμμου τμήματος που ορίζουν τα σημεία a και b.
Ορίζουμε την ακολουθία\\

\begin{align*}
    m_i & = \frac{a_i + b_i}{2} \\
    \intertext{όπου}
    a_i & = \left\{\begin{array}{lr}
        a, & \text{i = 0}\\
        a_{i - 1}, & i \geq 1 \wedge m_{i - 1} > m\\
        m_{i - 1}, & i \geq 1 \wedge m_{i - 1} \leq m
        \end{array}\right\} \\
    b_i & = \left\{\begin{array}{lr}
        b, & \text{i = 0}\\
        m_{i - 1}, & i \geq 1 \wedge m_{i - 1} > m\\
        b_{i - 1}, & i \geq 1 \wedge m_{i - 1} \leq m
        \end{array}\right\}
\end{align*}\\

Παρατηρούμε τα εξής:\\

$\bullet$ $a_i, b_i, m_i \in C \margin \forall i$\\

Αυτό αποδεικνύεται με μαθηματική επαγωγή στο i ως εξής:\\

ΒΑΣΗ ΕΠΑΓΩΓΗΣ: H πρόταση προς απόδειξη ισχύει για i = 0, καθώς\\

\begin{align*}
    a_0 & = a \in C && \text{ από υπόθεση} \\
    b_0 & = b \in C && \text{ από υπόθεση} \\
    m_0 & = \frac{a_0 + b_0}{2} = \frac{a + b}{2} \in C && \text{ λόγω midpoint covexivity}
\end{align*} \\

ΕΠΑΓΩΓΙΚΗ ΥΠΟΘΕΣΗ: Έστω ότι $a_i, b_i, m_i \in C \margin \forall i \leq n$ \\

ΕΠΑΓΩΓΙΚΟ ΒΗΜΑ: Θα δείξουμε ότι $a_{n + 1}, b_{n + 1}, m_{n + 1} \in C$ \\

Υποθέτουμε χωρίς βλάβη της γενικότητας ότι $m_n > m$. Έτσι έχουμε \\

\begin{align*}
    a_{n + 1} & = a_n \in C && \text{ από επαγωγική υπόθεση}\\
    b_{n + 1} & = m_n \in C && \text{ από επαγωγική υπόθεση} \\
    m_{n + 1} & = \frac{a_{n + 1} + b_{n + 1}}{2} = \frac{a_n + m_n}{2} \in C && \text{ λόγω midpoint covexivity}
\end{align*}

$\bullet$ Το σημείο m βρίσκεται πάντα επί του ευθύγραμμου τμήματος $\overlinesegment{a_{i}b_{i}}$,
γεγονός προφανές από τον ορισμό της ακολουθίας $m_i$.
Επομένως, ισχύει η ανισότητα $\norm{m_i - m} \leq \norm{a_i - b_i}$. \\

Θα αποδείξουμε με μαθηματική επαγωγή στο i, ότι $\norm{a_i - b_i} = \norm{a - b} \cdot 2^{-i}$\\

ΒΑΣΗ ΕΠΑΓΩΓΗΣ: Για i = 0, έχουμε: $\norm{a_0 - b_0} = \norm{a - b} = \norm{a - b} \cdot 2^{-0}$\\

ΕΠΑΓΩΓΙΚΗ ΥΠΟΘΕΣΗ: Έστω ότι $\norm{a_i - b_i} = \norm{a - b} \cdot 2^{-i} \margin \forall i \leq n$\\

ΕΠΑΓΩΓΙΚΟ ΒΗΜΑ: Υποθέτοντας χωρίς βλάβη της γενικότητας ότι $m_n > m$,
όπως και παραπάνω, για i = n + 1, έχουμε\\

\begin{align*}
    \norm{a_{n + 1} - b_{n + 1}} & = \norm{a_n - m_n} && \Leftrightarrow \\
    \norm{a_{n + 1} - b_{n + 1}} & = \norm{a_n - \frac{a_n + b_n}{2}} && \Leftrightarrow \\
    \norm{a_{n + 1} - b_{n + 1}} & = \norm{\frac{b_n - a_n}{2}} && \Leftrightarrow \\
    \norm{a_{n + 1} - b_{n + 1}} & = \norm{\frac{a_n - b_n}{2}} && \Leftrightarrow \\
    \norm{a_{n + 1} - b_{n + 1}} & = \norm{a_n - b_n} \cdot {2^{-1}} && \Leftrightarrow \\
    \norm{a_{n + 1} - b_{n + 1}} & = \norm{a - b} \cdot 2^{-n} \cdot {2^{-1}} && \Leftrightarrow \\
    \norm{a_{n + 1} - b_{n + 1}} & = \norm{a - b} \cdot 2^{-(n + 1)} && \Leftrightarrow
\end{align*}\\

Έτσι, έχουμε $0 \leq \norm{m_i - m} \leq \norm{x - y} \cdot 2^{-i}$\\

Είναι προφανές πως η ακολουθία $\norm{m_i - m}$ είναι φθίνουσα και λόγω του ότι είναι
και φραγμένη συγκλίνει στο μέγιστο κάτω φράγμα της, δηλαδή

\begin{align*}
    \lim_{i \to \infty} \norm{m_i - m} = 0 \Leftrightarrow \lim_{i \to \infty} m_i = m
\end{align*}

Τέλος, επειδή το σύνολο είναι κλειστό περιέχει τα όρια ακολουθιών στοιχείων του και αφού,
όπως δείξαμε $m_i \in C \margin \forall i$, συνεπάγεται ότι $m \in C$. Αυτό ισχύει για
οποιαδήποτε σημείο m επί του ευθύγραμμου τμήματος $\overlinesegment{αβ}$. Επομένως

\begin{align*}
    a, b \in C \implies m & = \theta_1 \cdot a + \theta_2 \cdot b \in C \margin \forall \theta_1, \theta_2 \in \R
\end{align*}

Επομένως, αν ένα σύνολο είναι κλειστό και midpoint \textit{convex}, τότε το σύνολο αυτό είναι \textit{convex}. 

\vspace{2in} %Leave more space for comments! \log(\frac{1}{\delta})

\pagebreak

\subsection*{5. Show that the \textit{convex hull} of a set S is the intersection of all \textit{convex} sets that contain
S. (The same method can be used to show that the conic, or \textit{affine}, or linear hull of a set S is
the intersection of all conic sets, or \textit{affine} sets, or subspaces that contain S.)}

Το \textit{convex hull} ενός συνόλου S, ορίζεται ως εξής:

\begin{align*}
    conv(S) & = \lbrace \sum_{i = 1}^{n} λ_i \cdot s_i
    \mid
    s_i \in S, \margin
    n \in \N, \margin
    λ_i \geq 0 \margin \forall i, \margin
    \sum_{i = 1}^{n} λ_i = 1
    \rbrace
\end{align*}

$\bullet$ Αρχικά θα αποδείξουμε ότι η τομή όλων των \textit{convex} υπερσυνόλων του $S$,
είναι υπερσύνολο του \textit{convex hull} του. \\

Έστω \textit{convex} σύνολο $C$, τέτοιο ώστε $C \supseteq S$.\\

Δεδομένου ότι το σύνολο $C$ είναι \textit{convex}, ισχύει

\begin{align*}
    \sum_{i = 1}^{n} λ_i * c_i \in C \text{ όπου }
    c_i \in C, \margin
    n \in \N, \margin
    λ_i \geq 0 \margin \forall i, \margin
    \sum_{i = 1}^{n} λ_i = 1
\end{align*}

Δεδομένου τώρα ότι $C \supseteq P$, ισχύει

\begin{align*}
    s_i \in S \Rightarrow s_i \in C \Rightarrow \sum_{i = 1}^{n} λ_i \cdot s_i \in C
\end{align*}

Ως εκ τούτου, προκύπτει $C \supseteq conv(S)$ και αφού αυτό ισχύει για οποιαδήποτε \textit{convex}
υπερσύνολο του $S$ συνεπάγεται πως και η τομή αυτών των συνόλων είναι υπερσύνολο του $conv(S)$.\\

$\bullet$ Στη συνέχεια θα δείξουμε ότι το \textit{convex hull} του συνόλου $S$ είναι υπερσύνολο της
τομής όλων των \textit{convex} υπερσυνόλων του.\\

Είναι προφανές πως $conv(S) \supseteq S$. Αρκεί να δείξουμε ότι το σύνολο $conv(S)$
είναι \textit{convex}, καθώς σε αυτή την περίπτωση θα ανήκει στο σύνολο \textit{convex} υπερσυνόλων του
$S$ και ως εκ τούτου θα αποτελεί υπερσύνολο της τομής τους.\\

Έστω δύο σημεία $x_1$ και $x_2$, τέτοια ώστε\\

\begin{align*}
    \sum_{i = 1}^{n} λ_i \cdot s_i = x_1 & \in conv(S) \\
    \sum_{i = 1}^{n} μ_i \cdot s_i = x_2 & \in conv(S)
\end{align*}\\

Έστω τώρα $\theta \in \lbrack 0, 1 \rbrack$\\

\begin{align*}
    \theta \cdot x_1 + (1 - \theta) \cdot x_2 & = \theta \cdot \sum_{i = 1}^{n} λ_i \cdot s_i + (1 - \theta) \cdot \sum_{i = 1}^{n} μ_i \cdot s_i && \Leftrightarrow \\
    \theta \cdot x_1 + (1 - \theta) \cdot x_2 & = \sum_{i = 1}^{n}(\theta \cdot λ_i + (1 - \theta) \cdot μ_i) \cdot s_i
\end{align*}\\

Δεδομένου ότι\\

\begin{align*}
    \sum_{i = 1}^{n}(\theta \cdot λ_i + (1 - \theta) \cdot μ_i) & = \theta \cdot \sum_{i = 1}^{n} λ_i + (1 - \theta) \cdot \sum_{i = 1}^{n} μ_i && \Leftrightarrow \\
    \sum_{i = 1}^{n}(\theta \cdot λ_i + (1 - \theta) \cdot μ_i) & = \theta \cdot 1 + (1 - \theta) \cdot 1 && \Leftrightarrow \\
    \sum_{i = 1}^{n}(\theta \cdot λ_i + (1 - \theta) \cdot μ_i) & = 1
\end{align*}\\

και\\

\begin{align*}
    \theta \cdot λ_i + (1 - \theta) \cdot μ_i \geq 0 \margin \forall i \in \lbrace 1, \dotsc, n \rbrace
\end{align*}

καταλήγουμε στο συμπέρασμα $\theta \cdot x_1 + (1 - \theta) \cdot x_2 \in conv(S)$
και άρα το σύνολο $conv(S)$ είναι \textit{convex} και ως εκ τούτου αποτελεί υπερσύνολο της
τομής των \textit{convex} υπερσυνόλων του $S$.\\

Αποδείξαμε λοιπόν ότι \\

\begin{align*}
    &\begin{rcases}
        conv(S) & \supseteq \bigcap C \text{ όπου C οποιοδήποτε \textit{convex} υπερσύνολο του S} \\
        conv(S) & \subseteqq \bigcap C \text{ όπου C οποιοδήποτε \textit{convex} υπερσύνολο του S}
    \end{rcases}
    \Rightarrow \\
    &conv(S) \equiv \bigcap C \text{ όπου C οποιοδήποτε \textit{convex} υπερσύνολο του S}
\end{align*}

\vspace{2in} %Leave more space for comments! \log(\frac{1}{\delta})

\pagebreak

\subsection*{6. What is the distance between two parallel hyperplanes ${\lbrace x \in \R^n \mid a^Tx = b_1 \rbrace}$ and
${\lbrace x \in \R^n \mid a^Tx = b_2 \rbrace}$ ?}

Τα δύο hyperplanes, έστω $B_1$ και $B_2$ είναι μεταξύ τους παράλληλα,
άρα έχει νόημα να μελετήσουμε την απόσταση μεταξύ τους.
Αν δεν ήταν παράλληλα, η απόσταση μεταξύ των $B_1$ και $B_2$ θα ήταν 0,
αφού θα υπήρχε σημείο τομής ανάμεσα τους. \\

Έστω ένα σημείο $x_1$ που ανήκει στο $B_1$ και έστω μία ευθεία Ε, 
η οποία διέρχεται από το $x_1$ και είναι κάθετη στο $Β_1$. \\

Η Ε τέμνει το $B_2$ στο σημείο $x_2$ επίσης κάθετα, 
γιατί τα δύο hyperplanes είναι μεταξύ τους παράλληλα. \\
 
Αρκεί να βρούμε την απόσταση μεταξύ των δύο σημείων $x_1$ και $x_2$, 
δηλαδη να βρούμε την $\norm{x_1-x_2}_2 $. \\

Η ευθεία Ε έχει ίδια κατεύθυνση με το α, 
γιατί και το α και η ευθεία Ε είναι κάθετα στα δύο hyperplanes, 
συνεπώς την ευθεία μπορούμε να την γράψουμε ως: $αy + x_1$. \\

Η τομή της Ε με το $Β_2$ είναι \\

$α^{Τ} \cdot x = b_2 \Leftrightarrow α^{Τ} \cdot (αy + x_1) = b_2 \Leftrightarrow 
α^{Τ}  \cdot  α  \cdot y + α^{Τ}  \cdot x_1 = b_2 \Leftrightarrow \\ 
\Leftrightarrow y = \frac{b_2 - α^{Τ} \cdot x_1}{α^{Τ} \cdot α} \Leftrightarrow 
y = \frac{b_2 - b_2}{α^{Τ} \cdot α} $ \\

Άρα \\

$x_2 = α \cdot y + x_1 \Leftrightarrow x_2 = a \cdot \frac{b_2 - b_1}{α^{Τ} \cdot α} + x_1$. \\

Συνεπώς \\

$\norm{x_1-x_2}_2 = 
\norm{x_1 - \frac{b_2 - b_1}{α^{Τ}} - x_1}_2 = 
\norm{-\frac{b_2 - b_1}{α^{Τ}}}_2 = 
\frac{|b_2 - b_1|}{\norm{α}_2}$

\vspace{2in} %Leave more space for comments! \log(\frac{1}{\delta})

\pagebreak

\subsection*{7. Let a and b be distinct points in $\R^n$ . Show that the set of all points that are closer (in
Euclidean Norm) to a than b is a halfspace.}

Έστω $S = \lbrace x \in \R^{n} \mid \norm{x - a}_2 \leq \norm{x - b}_2 \rbrace$ το σύνολο
των σημείων $x \in \R^{n}$, τα οποία βρίσκονται πιο κοντά στο a, σε σχέση με το b, 
βάσει Ευκλείδιας Νόρμας. Για τα σημεία αυτά ισχύει \\

\begin{align*}
    \norm{x - a}_2 & \leq \norm{x - b}_2 && \Leftrightarrow \\
    \norm{x - a}_2^2 & \leq \norm{x - b}_2^2 && \Leftrightarrow \\
    (x - a)^T \cdot (x - a) & \leq (x - b)^T \cdot (x - b) && \Leftrightarrow \\
    x^T \cdot x - x^T \cdot a - a^T \cdot x + a^T \cdot a & \leq x^T \cdot x - x^T \cdot b - b^T \cdot x + b^T \cdot b && \Leftrightarrow \\
    2 \cdot b^T \cdot x - 2 \cdot a^T \cdot x & \leq b^T \cdot b - a^T \cdot a && \Leftrightarrow \\
    (2 \cdot (b - a))^T \cdot x & \leq b^T \cdot b - a^T \cdot a
\end{align*}\\

Θέτοντας

\begin{align*}
    c & = 2 \cdot (b - a) \\
    d & = b^T \cdot b - a^T \cdot a
\end{align*}

λαμβάνουμε την ανισότητα $c^T \cdot x \leq d$, η οποία ορίζει έναν κλειστό halfspace.

\end{document}
