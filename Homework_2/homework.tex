
\documentclass[12pt]{article}

\usepackage[utf8]{inputenc}
\usepackage[greek, english]{babel}
\usepackage{alphabeta, amsfonts, amsmath, amssymb, amsthm, latexsym, stackrel, titlesec}

\newcommand{\R}{\mathbb{R}}
\newcommand{\N}{\mathbb{N}}
\newcommand{\norm}[1]{\left\lVert#1\right\rVert}
\newcommand{\margin}{\hspace{4pt}}
\newcommand{\centered}[1]{\begin{align*}#1\end{align*}}

\newenvironment{rcases}
	{\left.\begin{aligned}}
	{\end{aligned}\right\rbrace}

\setlength{\parindent}{0in}
\setlength{\oddsidemargin}{0in}
\setlength{\textwidth}{6.5in}
\setlength{\textheight}{10in}
\setlength{\topmargin}{-1.0in}
\setlength{\headheight}{18pt}

\titlespacing*{\subsection}
{0pt}{5.5ex plus 1ex minus .2ex}{4.3ex plus .2ex}

\title{\hugeΑλγοριθμική Επιχειρησιακή Έρευνα\\Δεύτερη Εργασία}
\author{Σιώρος Βασίλειος\\Ανδρινοπούλου Χριστίνα}
\date{Οκτώβριος 2019}

\begin{document}

\maketitle

\thispagestyle{empty}

\pagenumbering{arabic}



\pagebreak

\subsection*{1. Find a differentiable function f : R R such that f does not have an extremum at its
critical point.}

\vspace{2in}


\pagebreak

\subsection*{2. Given a positive integer S, which decompositions
a1 + + an = S
with the ai positive integers have the largest product a1 an?}

\vspace{2in}


\pagebreak

\subsection*{3. Find the optimal solution to the Diet Problem when the cost function is
Cost(x1, x2) = x1 + x2.}

\vspace{2in}


\pagebreak

\subsection*{4. Let A,B $\R^{n\times n}$. Show that the traditional way of computing their product AB requires
a total of $(2n-1)n^2$ arithmetic operations.}

Οι πίνακες Α και Β είναι τετραγωνικοί $(n\times n)$, δηλαδή αποτελόυνται από n γραμμές και από n στήλες, όπως φαίνεται παρακάτω: \\
$$A = \begin{bmatrix}
a_{11} & a_{22} & . . . . & a_{1n} \\
a_{21} & a_{22} & . . . . & a_{2n} \\
. & . & . . . . & . \\
a_{n1} & a_{n2} & . . . . & a_{nn}
\end{bmatrix}$$
$$B = \begin{bmatrix}
b_{11} & b_{22} & . . . . & b_{1n} \\
b_{21} & b_{22} & . . . . & b_{2n} \\
. & . & . . . . & . \\
b_{n1} & b_{n2} & . . . . & b_{nn}
\end{bmatrix}$$
\\
$$ Α\times B = 
\begin{bmatrix}
a_{11} & a_{22} & . . . . & a_{1n} \\
a_{21} & a_{22} & . . . . & a_{2n} \\
. & . & . . . . & . \\
a_{n1} & a_{n2} & . . . . & a_{nn}
\end{bmatrix} 
\begin{bmatrix}
b_{11} & b_{22} & . . . . & b_{1n} \\
b_{21} & b_{22} & . . . . & b_{2n} \\
. & . & . . . . & . \\
b_{n1} & b_{n2} & . . . . & b_{nn}
\end{bmatrix} =
\begin{bmatrix}
z_{11} & z_{22} & . . . . & z_{1n} \\
z_{21} & z_{22} & . . . . & z_{2n} \\
. & . & . . . . & . \\
z_{n1} & z_{n2} & . . . . & z_{nn}
\end{bmatrix} = Ζ
$$
\\
Για τον πολλαπλασιασμό των πινάκων Α και Β αρκεί να πολλαπλασιάσουμε την: \\ 
\[
\left.
\raisebox{1pt}[30pt]{\smash{$\begin{array}{r@{}l@{\,}l}
		\mbox{1η γραμμή του Α με την 1η στήλη του Β, για το $z_{11}$ $\rightarrow$ n γινόμενα και (n-1) προσθέσεις} \\
		\mbox{1η γραμμή του Α με την 2η στήλη του Β, για το $z_{12}$ $\rightarrow$ n γινόμενα και (n-1) προσθέσεις} \\
		\\
		\mbox{1η γραμμή του Α με την n στήλη του Β, για το $z_{1n}$ $\rightarrow$ n γινόμενα και (n-1) προσθέσεις} \\
		\end{array}$}}
\right\} n(n + (n-1))
\]
\\
\[
\left.
\raisebox{1pt}[30pt]{\smash{$\begin{array}{r@{}l@{\,}l}
		\mbox{2η γραμμή του Α με την 1η στήλη του Β, για το $z_{21}$ $\rightarrow$ n γινόμενα και (n-1) προσθέσεις} \\
		\mbox{2η γραμμή του Α με την 2η στήλη του Β, για το $z_{22}$ $\rightarrow$ n γινόμενα και (n-1) προσθέσεις} \\
	   	\\
		\mbox{2η γραμμή του Α με την n στήλη του Β, για το $z_{2n}$ $\rightarrow$ n γινόμενα και (n-1) προσθέσεις} \\
		\end{array}$}}
\right\} n(n + (n-1))
\]
\\
$$...$$
$$...$$
$$...$$
\\
\[
\left.
\raisebox{1pt}[30pt]{\smash{$\begin{array}{r@{}l@{\,}l}
		\mbox{n γραμμή του Α με την 1η στήλη του Β, για το $z_{n1}$ $\rightarrow$ n γινόμενα και (n-1) προσθέσεις} \\
		\mbox{n γραμμή του Α με την 2η στήλη του Β, για το $z_{n2}$ $\rightarrow$ n γινόμενα και (n-1) προσθέσεις} \\
		\\
		\mbox{n γραμμή του Α με την n στήλη του Β, για το $z_{nn}$ $\rightarrow$ n γινόμενα και (n-1) προσθέσεις} \\
		\end{array}$}}
\right\} n(n + (n-1))
\]
\\
Η κάθε γραμμή του πίνακα Α πολλαπλασιάζεται με όλες τις στήλες του Β και προκύπτει μία νέα γραμμή στον πίνακα Ζ. Η παραπάνω διαδικασία απαιτεί n(n+(n-1)) αριθμητικές παραστάσεις και επειδή αυτό θα συμβεί n φορές απαιτούνται συνολικά $n^2(n+(n-1)) = n^2(2n-1)$.
\vspace{2in}
\pagebreak

\subsection*{5. Consider the problem of solving a system of n linear equations in n unknowns. Show
that the Gaussian elimination method requires O(n3) arithmetic operations in order to either
compute a solution or to decide that no solution exist.}

\vspace{2in}


\pagebreak

\subsection*{6. Suppose that we are given a set of vectors in Rn that form a basis and let y be an
arbitrary vector in Rn. We wish to express y as a linear combination of the basis vectors. How
can this by accomplished?}

\vspace{2in}


\pagebreak

\subsection*{7. Study the paper with title: Do dogs know Calculus? found in the Readings folder.}

\vspace{2in}

\end{document}
