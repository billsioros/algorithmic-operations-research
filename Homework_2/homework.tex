
\documentclass[12pt]{article}

\usepackage[utf8]{inputenc}
\usepackage[greek, english]{babel}
\usepackage{alphabeta, amsfonts, amsmath, amssymb, amsthm, latexsym, stackrel, titlesec}

\newcommand{\R}{\mathbb{R}}
\newcommand{\N}{\mathbb{N}}
\newcommand{\norm}[1]{\left\lVert#1\right\rVert}
\newcommand{\margin}{\hspace{4pt}}
\newcommand{\centered}[1]{\begin{align*}#1\end{align*}}

\newenvironment{rcases}
	{\left.\begin{aligned}}
	{\end{aligned}\right\rbrace}

\setlength{\parindent}{0in}
\setlength{\oddsidemargin}{0in}
\setlength{\textwidth}{6.5in}
\setlength{\textheight}{10in}
\setlength{\topmargin}{-1.0in}
\setlength{\headheight}{18pt}

\titlespacing*{\subsection}
{0pt}{5.5ex plus 1ex minus .2ex}{4.3ex plus .2ex}

\title{\hugeΑλγοριθμική Επιχειρησιακή Έρευνα\\Δεύτερη Εργασία}
\author{Σιώρος Βασίλειος\\Ανδρινοπούλου Χριστίνα}
\date{Οκτώβριος 2019}

\begin{document}

\maketitle

\thispagestyle{empty}

\pagenumbering{arabic}



\pagebreak

\subsection*{1. Find a differentiable function f : R R such that f does not have an extremum at its
critical point.}

\vspace{2in}


\pagebreak

\subsection*{2. Given a positive integer S, which decompositions
a1 + + an = S
with the ai positive integers have the largest product a1 an?}

\vspace{2in}


\pagebreak

\subsection*{3. Find the optimal solution to the Diet Problem when the cost function is
Cost(x1, x2) = x1 + x2.}

\vspace{2in}


\pagebreak

\subsection*{4. Let A,B $\R^{n\times n}$. Show that the traditional way of computing their product AB requires
a total of $(2n-1)n^2$ arithmetic operations.}

Οι πίνακες Α και Β είναι τετραγωνικοί $(n\times n)$, δηλαδή αποτελόυνται από n γραμμές και από n στήλες, όπως φαίνεται παρακάτω: \\
$$A = \begin{bmatrix}
a_{11} & a_{22} & . . . . & a_{1n} \\
a_{21} & a_{22} & . . . . & a_{2n} \\
. & . & . . . . & . \\
a_{n1} & a_{n2} & . . . . & a_{nn}
\end{bmatrix}$$
$$B = \begin{bmatrix}
b_{11} & b_{22} & . . . . & b_{1n} \\
b_{21} & b_{22} & . . . . & b_{2n} \\
. & . & . . . . & . \\
b_{n1} & b_{n2} & . . . . & b_{nn}
\end{bmatrix}$$
\\
$$ Α\times B = 
\begin{bmatrix}
a_{11} & a_{22} & . . . . & a_{1n} \\
a_{21} & a_{22} & . . . . & a_{2n} \\
. & . & . . . . & . \\
a_{n1} & a_{n2} & . . . . & a_{nn}
\end{bmatrix} 
\begin{bmatrix}
b_{11} & b_{22} & . . . . & b_{1n} \\
b_{21} & b_{22} & . . . . & b_{2n} \\
. & . & . . . . & . \\
b_{n1} & b_{n2} & . . . . & b_{nn}
\end{bmatrix} =
\begin{bmatrix}
z_{11} & z_{22} & . . . . & z_{1n} \\
z_{21} & z_{22} & . . . . & z_{2n} \\
. & . & . . . . & . \\
z_{n1} & z_{n2} & . . . . & z_{nn}
\end{bmatrix} = Ζ
$$
\\
Για τον πολλαπλασιασμό των πινάκων Α και Β αρκεί να πολλαπλασιάσουμε την: \\ 
\[
\left.
\raisebox{1pt}[30pt]{\smash{$\begin{array}{r@{}l@{\,}l}
		\mbox{1η γραμμή του Α με την 1η στήλη του Β, για το $z_{11}$ $\rightarrow$ n γινόμενα και (n-1) προσθέσεις} \\
		\mbox{1η γραμμή του Α με την 2η στήλη του Β, για το $z_{12}$ $\rightarrow$ n γινόμενα και (n-1) προσθέσεις} \\
		\\
		\mbox{1η γραμμή του Α με την n στήλη του Β, για το $z_{1n}$ $\rightarrow$ n γινόμενα και (n-1) προσθέσεις} \\
		\end{array}$}}
\right\} n(n + (n-1))
\]
\\
\[
\left.
\raisebox{1pt}[30pt]{\smash{$\begin{array}{r@{}l@{\,}l}
		\mbox{2η γραμμή του Α με την 1η στήλη του Β, για το $z_{21}$ $\rightarrow$ n γινόμενα και (n-1) προσθέσεις} \\
		\mbox{2η γραμμή του Α με την 2η στήλη του Β, για το $z_{22}$ $\rightarrow$ n γινόμενα και (n-1) προσθέσεις} \\
	   	\\
		\mbox{2η γραμμή του Α με την n στήλη του Β, για το $z_{2n}$ $\rightarrow$ n γινόμενα και (n-1) προσθέσεις} \\
		\end{array}$}}
\right\} n(n + (n-1))
\]
\\
$$...$$
$$...$$
$$...$$
\\
\[
\left.
\raisebox{1pt}[30pt]{\smash{$\begin{array}{r@{}l@{\,}l}
		\mbox{n γραμμή του Α με την 1η στήλη του Β, για το $z_{n1}$ $\rightarrow$ n γινόμενα και (n-1) προσθέσεις} \\
		\mbox{n γραμμή του Α με την 2η στήλη του Β, για το $z_{n2}$ $\rightarrow$ n γινόμενα και (n-1) προσθέσεις} \\
		\\
		\mbox{n γραμμή του Α με την n στήλη του Β, για το $z_{nn}$ $\rightarrow$ n γινόμενα και (n-1) προσθέσεις} \\
		\end{array}$}}
\right\} n(n + (n-1))
\]
\\
Η κάθε γραμμή του πίνακα Α πολλαπλασιάζεται με όλες τις στήλες του Β και προκύπτει μία νέα γραμμή στον πίνακα Ζ. Η παραπάνω διαδικασία απαιτεί n(n+(n-1)) αριθμητικές παραστάσεις και επειδή αυτό θα συμβεί n φορές απαιτούνται συνολικά $n^2(n+(n-1)) = n^2(2n-1)$.
\vspace{2in}
\pagebreak

\subsection*{5. Consider the problem of solving a system of n linear equations in n unknowns. Show
that the Gaussian elimination method requires O($n^3$) arithmetic operations in order to either
compute a solution or to decide that no solution exist.}

Η μέθοδος απαλοιφής του Gauss είναι μία μέθοδος για την επίλυση πυκνών γραμμικών συστημάτων, δηλαδή συστημάτων που ο πίνακας των συντελεστών των αγνώστων στοιχείων αποτελείται κυρίως από μη μηδενικά στοιχεία. \\
Πριν προχωρήσουμε στην εύρεση της πολυπλοκότητας της μεθόδου, θα εξηγήσουμε πως δουλεύει η μέθοδος. \\ \\
Στόχος της απαλοιφής του Gauss είναι να λύσουμε ένα γραμμικό σύστημα $Ax = b$, όπου $A\epsilon \R^{n\times n}$, $x\epsilon \R^{n\times 1}$ και $B\epsilon \R^{n\times 1}$. Για να λύσουμε το σύστημα αυτό, η μέθοδος απαλοιφής του Gauss προχωρά σε τριγωνοποίηση του πίνακα A, κι έτσι ο πίνακας γίνεται άνω τριγωνικός, κι έπειτα με προς τα πίσω αντικατάσταση βρίσκουμε τους αγνώστους. \\ \\
Πιο αναλυτικά, έστω ότι βρισκόμαστε στο k-οστό βήμα της μεθόδου (αυτό σημαίνει ότι κοιτάμε την k γραμμή του πίνακα A). Τα βήματα που ακολουθούμε εδώ είναι: \\
1. Βρίσκουμε $(n-k)$ το πλήθος πολλαπλασιαστές m, όπου $m_{ik} = - \frac{a_{ik}}{a_{kk}}$ για $i=k+1,k+2,...,n$. \\
2. Εφαρμόζουμε τον τύπο $a_{ij}^{new} = a_{ij}^{old} + m_{ik}\times a_{kj}^{(k)}$ \\ για $i = k+1, k+2, ...,n$ και $j = k, k+1,...,n$ \\
3. Εφαρμόζουμε τον τύπο $b_{i}^{new} = b_{i}^{old} + m_{ik}\times _{k}^{(k)}$ \\ για $i = k+1, k+2, ...,n$ \\ \\
Για την εύρεση της πολυπλοκότητας θα αναφερθούμε αρχικά στη γενική περίπτωση επίλυσης $l$ το πλήθος γραμμικών συστημάτων με τον ίδιο πίνακα αγνώστων Α. Άλλωστε, η "βαρια" υπολογιστικά εργασία είναι η τριγωνοποίηση του πίνακα Α. Έπειτα θα περάσουμε στην περίπτωση όπου το $l=1$. \\
Έστω πάλι ότι βρισκόμαστε στο k βήμα της μεθόδου:
$$
\begin{bmatrix}
a_{11} & a_{12} & a_{13} & ... & a_{1k} & a_{1,k+1} & ... & a_{1n}\\
& a_{22}  & a_{23}  & ... & a_{2k}  & a_{2,k+1}  & ... & a_{2n} \\
&  & a_{33}  & ... & a_{3k}  & a_{3,k+1}  & ... & a_{3n} \\
&  &  &  & . & . &  & .\\
& 0 &  & \ddots  & . & . &  & .\\
&  &  &  & . & . &  & .\\
&  &  &  & a_{kk}  & a_{k,k+1}  & ... & a_{kn}  \\
&  &  &  & a_{k+1,k}  & a_{k+1,k+1}  & ... & a_{k+1,n} \\
&  &  &  & . & . &  & .\\
& 0 &  &  & . & . &  & .\\
&  &  &  & . & . &  & .\\
&  &  &  & a_{nk}  & a_{n,k+1}  & ... & a_{nn} 
\end{bmatrix} 
\begin{bmatrix}
x_{1}^{(1)} & x_{1}^{(2)} & ... & x_{1}^{(l)}\\
x_{2}^{(1)} & x_{2}^{(2)} & ... & x_{2}^{(l)}\\
x_{3}^{(1)} & x_{3}^{(2)} & ... & x_{3}^{(l)}\\
. & . &  & .\\
. & . &  & .\\
. & . &  & .\\
x_{k}^{(1)} & x_{k}^{(2)} & ... & x_{k}^{(l)}\\
x_{k+1}^{(1)} & x_{k+1}^{(2)} & ... & x_{k+1}^{(l)}\\
. & . &  & .\\
. & . &  & .\\
. & . &  & .\\
x_{n}^{(1)} & x_{n}^{(2)} & ... & x_{n}^{(l)}
\end{bmatrix}  
=
\begin{bmatrix}
b_{1}^{(1)} & b_{1}^{(2)} & ... & b_{1}^{(l)}\\
b_{2}^{(1)} & b_{2}^{(2)} & ... & b_{2}^{(l)}\\
b_{3}^{(1)} & b_{3}^{(2)} & ... & b_{3}^{(l)}\\
. & . &  & .\\
. & . &  & .\\
. & . &  & .\\
b_{k}^{(1)} & b_{k}^{(2)} & ... & b_{k}^{(l)}\\
b_{k+1}^{(1)} & b_{k+1}^{(2)} & ... & b_{k+1}^{(l)}\\
. & . &  & .\\
. & . &  & .\\
. & . &  & .\\
b_{n}^{(1)} & b_{n}^{(2)} & ... & b_{n}^{(l)}
\end{bmatrix}
$$

Για να εκτελέσουμε το βήμα 1 και να καταφέρουμε να μηδενίσουμε τα $a_{k+1,k},...,a_{nk}$, πρέπει να βρόυμε $(n-k)$ το πλήθος πολλαπλασιαστές τύπου m. Αρα, σε αυτό το βήμα θα εκτελεστούν $(n-k)$ διαιρέσεις. \\
Για να εκτελεστεί τα βήματα 2 και 3: \\
-για το γινόμενο $ma$ θα εκτελεστούν $(n-k+l)$ γινόμενα $(n-k)$ φορές    \\
-για την πρόσθεση $a+ma$ θα εκτελεστούν $(n-k+l)$ προσθαφαιρέσεις $(n-k)$ φορές. \\ \\
Συνεπώς, για την πλήρη εκτέλεση του αλγορίθμου θα χρειαστούν:
$$ \sum_{k=1}^{n-1} {(n-k)}	\mbox{	διαιρέσεις}$$
$$\sum_{k=1}^{n-1} {(n-k)(n-k+l)} \mbox{	πολλαπλασιασμοί} $$ 
$$\sum_{k=1}^{n-1} {(n-k)(n-k+l)} \mbox{	προσθαφαιρέσεις} $$  
Κάνοντας χρήση των: \\
$$
\sum_{k=1}^{m}k = \frac{m(m+1)}{2} \mbox{ και } \sum_{k=1}^{m}k^2
$$
προκύπτει τελικά ότι για την τριγωνοποίηση του πίνακα Α χρειαζόμαστε:
$$ \sum_{k=1}^{n-1} {(n-k)} = \sum_{k=1}^{n-1}n - \sum_{k=1}^{n-1}k = $$
$$ = n(n-1) - \frac{(n-1)(n-1+1)}{2} = \frac{n(n-1)}{2} - \frac{n(n-1)}{2} = \frac{n(n-1)}{2}	\mbox{	διαιρέσεις}$$ \\ \\
$$\sum_{k=1}^{n-1} {(n-k)(n-k+l)} = \sum_{k=1}^{n-1}{(n^2-nk+nl-kn+k^2-kl)} = $$
$$ = \sum_{k=1}^{n-1}n^2 - 2\sum_{k=1}^{n-1}{nk} + \sum_{k=1}^{n-1}{nl} + \sum_{k=1}^{n-1}k^2 - \sum_{k=1}^{n-1}{kl} = $$
$$ = n^2\sum_{k=1}^{n-1}1 -2n\sum_{k=1}^{n-1}k +nl\sum_{k=1}^{n-1}1 + \sum_{k=1}^{n-1}k^2 - l\sum_{k=1}^{n-1}k = $$
$$ = n^2(n-1) - 2n\frac{(n-1)n}{2} + nl(n-1) + \frac{(n-1)n(2n-1)}{6} - l\frac{(n-1)n}{2} = $$
$$ = n^2(n-1) - n^2(n-1) + \frac{1}{2}nl(n-1) + \frac{n(n-1)(2n-1)}{6} = $$
$$ = \frac{nl(n-1)}{2} + \frac{n(n-1)(2n-1)}{6} = \frac{3nl(n-1)}{6} + \frac{n(n-1)(2n-1)}{6} = $$ $$ = \frac{n(n-1)(3l+2n-1)}{6} \mbox{	πολλαπλασιασμοί} $$ \\ \\ 
$$\sum_{k=1}^{n-1} {(n-k)(n-k+l)} = \sum_{k=1}^{n-1}{(n^2-nk+nl-kn+k^2-kl)} = $$
$$ = \sum_{k=1}^{n-1}n^2 - 2\sum_{k=1}^{n-1}{nk} + \sum_{k=1}^{n-1}{nl} + \sum_{k=1}^{n-1}k^2 - \sum_{k=1}^{n-1}{kl} = $$
$$ = n^2\sum_{k=1}^{n-1}1 -2n\sum_{k=1}^{n-1}k +nl\sum_{k=1}^{n-1}1 + \sum_{k=1}^{n-1}k^2 - l\sum_{k=1}^{n-1}k = $$
$$ = n^2(n-1) - 2n\frac{(n-1)n}{2} + nl(n-1) + \frac{(n-1)n(2n-1)}{6} - l\frac{(n-1)n}{2} = $$
$$ = n^2(n-1) - n^2(n-1) + \frac{1}{2}nl(n-1) + \frac{n(n-1)(2n-1)}{6} = $$
$$ = \frac{nl(n-1)}{2} + \frac{n(n-1)(2n-1)}{6} = \frac{3nl(n-1)}{6} + \frac{n(n-1)(2n-1)}{6} = $$ $$ = \frac{n(n-1)(3l+2n-1)}{6} \mbox{	προσθαφαιρέσεις} $$ \\ \\ 
\\ ενώ για τον υπολογισμό των x απαιτούνται:
$$ l\sum_{k=1}^{n}1 \mbox{ διαιρέσεις}$$
$$ l\sum_{k=1}^{n-1}{(n-k)} \mbox{ πολλαπλασιασμοί}$$
$$ l\sum_{k=1}^{n-1}{(n-k)} \mbox{ προσθαφαιρέσεις}$$
Δηλαδή:
$$ nl \mbox{ διαιρέσεις} $$
$$ \frac{n(n-1)l}{2} \mbox{ πολλαπλασιασμοί}$$
$$ \frac{n(n-1)l}{2} \mbox{ προσθαφαιρέσεις}$$
Συνεπώς, το σύνολο των πράξεων είναι:
$$ \frac{n(n-1+2l)}{2} \mbox{ διαιρέσεις} $$
$$ \frac{n(n-1)(2n-1+6l)}{6} \mbox{ πολλαπλασιασμοί}$$
$$ \frac{n(n-1)(2n-1+6l)}{6} \mbox{ προσθαφαιρέσεις}$$
Αν $l=1$ απαιτούνται:
$$ \frac{n^2}{2} + \frac{n}{2} \mbox{ διαιρέσεις}$$
$$ \frac{n^3}{3} + \frac{n^2}{2} - \frac{5n}{6} \mbox{ πολλαπλασιασμοί}$$
$$ \frac{n^3}{3} + \frac{n^2}{2} - \frac{5n}{6} \mbox{ προσθαφαιρέσεις}$$
Άρα, η μέθοδος απαλοιφής του Gauss για την επίλυση γραμμικών συστημάτων απαιτεί Ο($n^3$).
\vspace{2in}


\pagebreak

\subsection*{6. Suppose that we are given a set of vectors in $R^n$ that form a basis and let y be an
arbitrary vector in $R^n$. We wish to express y as a linear combination of the basis vectors. How
can this by accomplished?}

Έστω ότι τα $\vec{b_1}, \vec{b_2},...,\vec{b_n}$ είναι το σύνολο που αποτελόυν τη βάση. \\
Βάση ενός διανυσματικού χώρου V είναι ένα σύνολο γραμμικώς ανεξάρτητων διανυσμάτων που παράγουν το V. Με άλλα λόγια το σύνολο $B=\{\vec{b_1}, \vec{b_2},...,\vec{b_n}\}$ είναι βάση του V αν: \\
1. $\vec{b_1}, \vec{b_2},...,\vec{b_n} \epsilon V$ \\
2. $\vec{b_1}, \vec{b_2},...,\vec{b_n}$ είναι γραμμικώς ανεξάρτητα. (τα $\vec{b_1}, \vec{b_2},...,\vec{b_n}$ είναι γραμμικώς ανεξάρτητα ανν η σχέση $λ_1\vec{b_1} + λ_2\vec{b_2} + ... + λ_n\vec{b_n} = \vec{0}$ ισχύει μόνο για $λ_1=λ_2=...=λ_n = 0$) \\
3. $\forall b \epsilon V$, $\exists λ_1,λ_2,...,λ_n \epsilon \R$ τω $b = λ_1\vec{b_1} + λ_2\vec{b_2} + ... + λ_n\vec{b_n}$. \\ \\
Για να εκφράσουμε ένα τυχαίο διάνυσμα $\vec{y}\epsilon \R^n$ πρέπει να βρούμε κατάλληλους συντελεστές τω 
$$ y_j = \sum_{i=1}^{n}{a_ib_j^i} \mbox{     } \forall j \leq n $$ όπου το j δηλώνει το j-οστό στοιχείο των διανυσμάτων $\vec{y}$ και $\vec{b}$ και $a_i \epsilon \R$ \\ \\
Θα αποδείξουμε ότι οι παραπάνω εξισώσεις έχουν μοναδική λύση. \\
Έστω διάνυσμα $\vec{z} \epsilon \R^n$ και έστω ότι μπορεί να αναπαρασταθεί με δύο διαφορετικούς τρόπους: \\
$$ \vec{z} = d_1\vec{b_1} + d_2\vec{b_2} + ... + d_n\vec{b_n} (1)$$
$$ \vec{z} = m_1\vec{b_1} + m_2\vec{b_2} + ... + m_n\vec{b_n} (2)$$
με $d_i,m_i \epsilon \R$. Επειδή, όμως, (1) = (2) έχουμε: 
$$ d_1\vec{b_1} + d_2\vec{b_2} + ... + d_n\vec{b_n} = m_1\vec{b_1} + m_2\vec{b_2} + ... + m_n\vec{b_n} \Leftrightarrow$$
$$ (d_1-m_1)\vec{b_1} + (d_2-m_2)\vec{b_2} + ... + (d_n-m_n)\vec{b_n} = 0 $$ 
κι επειδή τα $\vec{b_1}, \vec{b_2},...,\vec{b_n}$ είναι διανύσματα βάσης, είναι γραμμικώς ανεξάρτητα, άρα: \\
$$ (d_1-m_1) = 0, (d_2-m_2) = 0, ... , (d_n-m_n) = 0 $$ 
$$ d_1 = m_1, d_2 = m_2, ...,d_n = m_n $$
Συνεπώς, οι δύο αναπαραστάσεις του διανύσματος $\vec{z}$ είναι ίδιες και η αναπαράστασή του από διανύσματα βάσης είναι μοναδική. \\  \\
Για να βρούμε τα μοναδικά $a_i$ για να εκφράσουμε ένα τυχαίο διάνυσμα $\vec{y}$ ως γραμμικό συνδυασμό των διανυσμάτων βάσης καταλήγουμε σε μια σχέση όπως αυτή: \\
$$\vec{y} = a_1\vec{b_1} + a_2\vec{b_2} + ... + a_n\vec{b_n} \Leftrightarrow $$
$$ \begin{bmatrix}
	y_{1}\\
	y_{2}\\
	.\\
	.\\
	.\\
	y_{n}
\end{bmatrix} = a_1
\begin{bmatrix}
b_{1}^{(1)}\\
b_{2}^{(1)}\\
.\\
.\\
.\\
n_{n}^{(1)}
\end{bmatrix} \ + a_2
\begin{bmatrix}
b_{1}^{(2)}\\
b_{2}^{(2)}\\
.\\
.\\
.\\
n_{n}^{(2)}
\end{bmatrix} \ +...+a_n
\begin{bmatrix}
b_{1}^{(n)}\\
b_{2}^{(n)}\\
.\\
.\\
.\\
n_{n}^{(n)}
\end{bmatrix} \Leftrightarrow \ $$
$$ y_1 = a_1b_{1}^{(1)} + a_2b_{1}^{(2)} +...+ a_nb_{1}^{(n)} $$
$$ y_2 = a_1b_{2}^{(1)} + a_2b_{2}^{(2)} +...+ a_nb_{2}^{(n)} $$
$$ ... $$
$$ y_n = a_1b_{n}^{(1)} + a_2b_{n}^{(2)} +...+ a_nb_{n}^{(n)} $$
Άρα, για την εύρεση των μοναδικών $a_i$ μπορούμε να ελαχιστοποιήσουμε μία συνάρτηση, που υπόκειται σε αυτούς τους περιορισμούς. \\
\vspace{2in}


\pagebreak

\subsection*{7. Study the paper with title: Do dogs know Calculus? found in the Readings folder.}

\vspace{2in}

\end{document}
