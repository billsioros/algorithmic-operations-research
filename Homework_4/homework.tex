
\documentclass[12pt]{article}

\usepackage[utf8]{inputenc}
\usepackage[greek, english]{babel}

% Packages
\usepackage{alphabeta}
\usepackage{amsmath}
\usepackage{amsthm}
\usepackage{caption}
\usepackage{color}
\usepackage{fullpage}
\usepackage{graphicx}
\usepackage{latexsym}
\usepackage{listings}
\usepackage{pxfonts}
\usepackage{stackrel}
\usepackage{titlesec}
\usepackage{subfig}
\usepackage{tikz}
\usepackage{float}
\usepackage{hyperref}

% Commands
\newcommand{\N}{\mathbb{N}}
\newcommand{\R}{\mathbb{R}}
\newcommand{\code}[2]{\lstinputlisting[caption={#2}]{#1}}
\newcommand{\margin}{\hspace{4pt}}
\newcommand{\norm}[1]{\left\lVert#1\right\rVert}
\newcommand{\abs}[1]{\left\lvert#1\right\rvert}

% Environments
\newenvironment{matlab}
	{\begin{figure}[hp]\centering\captionsetup{justification=centering}}
	{\end{figure}}

\newenvironment{rcases}
	{\left.\begin{aligned}}
	{\end{aligned}\right\rbrace}

% Python Syntax Highlighting
\definecolor{string_color}{RGB}{0, 161, 13}
\definecolor{comment_color}{RGB}{46, 46, 46}
\definecolor{keyword_color}{RGB}{0, 112, 191}
\definecolor{background_color}{RGB}{250, 250, 250}

\lstset{
    framesep=15pt,
    xleftmargin=15pt,
    xrightmargin=15pt,
    language=Python,
    captionpos=b,
    numbers=right,
    numberstyle=\small\ttfamily,
    frame=lines,
    showspaces=false,
    showtabs=false,
    breaklines=true,
    showstringspaces=false,
    breakatwhitespace=true,
    commentstyle=\color{comment_color}\textit,
    keywordstyle=\bfseries\color{keyword_color}\textbf,
    stringstyle=\color{string_color}\textit,
    morekeywords={self, lambda, __init__, __del__, __name__, for, in, not, and, or, :},
    basicstyle=\small\ttfamily,
    tabsize=4,
    keepspaces=true,
    columns=flexible,
    backgroundcolor=\color{background_color}
}

% Links
\hypersetup{
    colorlinks=true,
    linkcolor=blue,
    filecolor=magenta,
    urlcolor=cyan,
}

% Lengths
\setlength{\parindent}{0in}
\setlength{\oddsidemargin}{0in}
\setlength{\textwidth}{6.5in}
\setlength{\textheight}{10in}
\setlength{\topmargin}{-1.0in}
\setlength{\headheight}{18pt}

\titlespacing*{\subsection}
{0pt}{5.5ex plus 1ex minus .2ex}{4.3ex plus .2ex}

\title{\hugeΑλγοριθμική Επιχειρησιακή Έρευνα\\Τέταρτη Εργασία}
\author{Σιώρος Βασίλειος\\Ανδρινοπούλου Χριστίνα}
\date{Νοέμβριος 2019}

\begin{document}

\maketitle

\pagenumbering{gobble}

\pagebreak


\subsection*{1. Consider the problem
\begin{align*}
    \min_{x_1, x_2} \{2 \cdot x_1 & + 3 \cdot \abs{x_2 - 10}\} \\
    &s.t. \\
    \abs{x_1 + 2} & + \abs{x_2} \leq 5
\end{align*}
and reformulate it as a linear programming problem.}

Γνωρίζουμε ότι

\begin{align*}
    \abs{x} & \geq +x \margin \forall x \in \R \\
    \abs{x} & \geq -x \margin \forall x \in \R
\end{align*}

Θέτουμε λοιπόν

\begin{align*}
    a_1 & = \abs{x_2 - 10} \\
    a_2 & = \abs{x_1 + 2} \\
    a_3 & = \abs{x_2}
\end{align*}

και προσθέτουμε τους περιορισμούς

\begin{align*}
    a_1 & \geq +x_2 - 10 \\
    a_1 & \geq -x_2 + 10 \\
    a_2 & \geq +x_1 + 2 \\
    a_2 & \geq -x_1 - 2 \\
    a_3 & \geq +x_2 \\
    a_3 & \geq -x_2
\end{align*}

Έτσι, το δοθέν πρόβλημα λαμβάνει την ακόλουθη ισοδύναμη μορφή

\begin{align*}
    \min_{x_1, x_2, a_1, a_2, a_3} & \{2 \cdot x_1 + 3 \cdot a_1\} \\
    &s.t. \\
    a_1 & \geq +x_2 - 10 \\
    a_1 & \geq -x_2 + 10 \\
    a_2 & \geq +x_1 + 2 \\
    a_2 & \geq -x_1 - 2 \\
    a_3 & \geq +x_2 \\
    a_3 & \geq -x_2 \\
    a_2 + a_3 & \leq 5
\end{align*}

\vspace{2in}

\pagebreak

\subsection*{2. (Road lighting) Consider a road divided in n segments that is illuminated by m
	lamps. Let \(p_{j}\) be the power of the jth lamp. The illumination \(I_{i}\) of the ith segment is assumed
	to be \(\sum_{j=1}^{m}{a_{ij} \cdot p_{j}}\)
	where \(a_{ij}\) are known coefficients. Let \(I_{i}^{*}\) be the desired illumination of road i.
	We are interested in choosing the lamp powers \(p_{j}\) so that the illuminations \(I_{i}\) are close to the
	desired \(I_{i}^{*}\). Provide a reasonable linear programming formulation of this problem.}

Έστω δρόμος χωρισμένος σε n κομμάτια. \\

Για να καταφέρουμε στο i κομμάτι του δρόμου ο υπάρχον φωτεισμός \(I_{i}\) να είναι όσο το δυνατόν πιο κοντά στον επιθυμητό φωτισμό \(I_{i}^{*}\) πρέπει η διαφορά των δύο φωτεισμών να είναι η μικρότερη δυνατή. \\

Συνεπώς, πρεπει να διαλέξουμε:
\[ \mbox{min }\abs{ I_{i}^{*} - I_{i} } \mbox{ για } 1 \leq i \leq n \]

Συνολικά, για ολόκληρο τον δρόμο πρέπει: \\
\[ \mbox{min } \sum_{i=1}^{n}{\abs{ I_{i}^{*} - I_{i} }}  \]
κι επειδή \(I_{i} = \sum_{j=1}^{m}{a_{ij} \cdot p_{j}}\) η παραπάνω σχέση μπορεί να γραφεί και ως \\
\[ \mbox{min } \sum_{i=1}^{n}{\abs{ I_{i}^{*} - \sum_{j=1}^{m}{a_{ij} \cdot p_{j}} }}  \]
Η μεταβλητή που ψάχνουμε είναι τα \(p_{j}\). \\

Οι περιορισμοί είναι (για προφανείς λόγους): \\
\[ p_{j} \geq 0 \mbox{για } 1 \leq j \leq m \]

\vspace{2in}

\pagebreak

\subsection*{3. Consider a school district with I neighborhoods, J schools and G grades
	at each school. Each school j has a capacity \(C_{jg}\) for grade g. In each neighborhood i, the
	student population of grade i is \(S_{ig}\) . Finally the distance of school j from neighborhood i is \(d_{ij}\) .
	Formulate a linear programming problem whose objective is to assign all students to schools,
	while minimizing the total distance traveled by all students. (You may ignore the fact that
	numbers of students must be integer).}

Στόχος είναι να ελαχιστοποιηθεί η απόσταση που διανύουν οι μαθητές για να φτάσουν στο σχολείο τους. \\

Οι περιορισμοί εδώ είναι:

\begin{itemize}
    \item κάθε σχολείο μπορεί να δεχτεί συγκεκριμένο αριθμό μαθητών.
    \item όλα τα παιδιά πρέπει να πηγαίνουν στο σχολείο.
\end{itemize}


Η κωδικοποίηση των απαραίτητων μεγεθών για το πρόβλημα είναι η εξής:

\begin{itemize}
    \item \( I \mbox{ γειτονιές}\)
    \item \( J \mbox{ σχολέια}\)
    \item \( G \mbox{ τάξεις}\)
    \item \( C_{jg} \mbox{ χωρητικότητα μαθητών στο σχολείο j στην τάξη g}\)
    \item \( S_{ig} \mbox{ πληθυσμός παιδιών στη γειτονιά i στην τάξη g}\)
    \item \( d_{ij} \mbox{ απόσταση γειτονιάς i από σχολείο j}\)
    \item \( k_{igj} \mbox{ οι μαθητές τάξης g που μένουν στη γειτονιά i και πηγαίνουν στο σχολείο j}\)
\end{itemize}

Θέλουμε να μειωθεί η συνολική απόσταση που διανύουν οι μαθητές, άρα: \\
\[ \mbox{min } \sum_{\forall i \in I}{ \sum_{\forall j \in J} { \sum_{\forall g \in G} {k_{igj} \cdot d_{ij} } }  }     \]

Υπό τους περιορισμούς:

\begin{itemize}
    \item το σχολείο j μπορεί να δεχτεί μέχρι \( C_{jg} \) μαθητές στην τάξη g
    \[  \sum_{\forall i \in I}{ k_{igj} \leq C_{jg} } \mbox{ } \forall g \in G \mbox{ } \forall j \in J   \]
    \item Κάθε παιδί πρέπει να πηγαίνει σχολείο:
    \[  \sum_{\forall j \in J}{ k_{igj} = S_{ig} } \mbox{ } \forall g \in G \mbox{ } \forall i \in I   \]
    \item Δε μπορεί η αντιστοίχιση μαθητών σε σχολεία να είναι αρνητική:
    \[  k_{igj} \geq 0  \]
\end{itemize}

\vspace{2in}

\pagebreak

\subsection*{4. Consider a set P described by linear inequality constraints
\[ P = \{x \in \R^n : a^{T}_{i} \cdot x \leq b_i, \margin i = 1, \dotsc, m\} \]
A ball with center y and radius r is defined as the set of all points within distance r from y.
We are interested in finding a ball with the largest possible radius, which is entirely contained
within the set \( P \). Provide a linear programming formulation of this problem.}

Μια σφαίρα, η οποία βρίσκεται εξ ολοκλήρου εντός του συνόλου

\[ P = \{x \in \R^n : a^{T}_{i} \cdot x \leq b_i, \margin i = 1, \dotsc, m\} \]

πρέπει να συμμορφώνεται στους εξής περιορισμούς:

\begin{enumerate}
    \item Το κέντρο της σφαίρας πρέπει να ανήκει στο συνόλο \( P \).
    \item Η απόσταση, του κέντρου της σφαίρας από οποιοδήποτε ύπερεπίπεδο, αντιστοιχεί σε μία
    από τις εξισώσεις \( a^{T}_{i} \cdot x = b_i, \margin i = 1, \dotsc, m\),
    που περιγράφουν το σύνολο \( P \), πρέπει να είναι τουλάχιστον ίση με \( r \).
\end{enumerate}

Η απόσταση, μεταξύ ενός σημείου \( p \) και ενός ύπερεπιπέδου \( κ \cdot x + λ \),
δίνεται από την έκφραση

\[ \frac{\abs{κ \cdot p + λ}}{\norm{κ}} \]

Λοιπόν, μπορούμε να εκφράσουμε μαθηματικά τους παραπάνω περιορισμούς ως εξής:

\begin{align*}
    a^{T}_{i} \cdot c & \leq b_i, \margin i = 1, \dotsc, m && (1)\\
    \frac{\abs{a^{T}_{i} \cdot c - b_i}}{\norm{a^{T}_{i}}} \geq r
    \stackrel{(1)}{\Rightarrow}\frac{b_i - a^{T}_{i} \cdot c}{\norm{a^{T}_{i}}} & \geq r, \margin i = 1, \dotsc, m&& (2)
\end{align*}

Τέλος, ως αντικειμενική συνάρτηση επιλέγουμε τη σχέση

\[ \max_{c, r} r \]

Συνοψίζοντας, το δοθέν πρόβλημα εκφράζεται ως πρόβλημα γραμμικου προγραμματισμού ως εξής:

\[ \max_{c, r} r \]

\[ s.t. \]

\begin{align*}
    a^{T}_{i} \cdot c & \leq b_i, \margin i = 1, \dotsc, m \\
    \frac{b_i - a^{T}_{i} \cdot c}{\norm{a^{T}_{i}}} & \geq r, \margin i = 1, \dotsc, m
\end{align*}

\pagebreak

\end{document}
