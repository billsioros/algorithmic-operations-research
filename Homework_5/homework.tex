
\documentclass[12pt]{article}

\usepackage[utf8]{inputenc}
\usepackage[greek, english]{babel}

% Packages
\usepackage{alphabeta}
\usepackage{amsmath}
\usepackage{amsthm}
\usepackage{caption}
\usepackage{color}
\usepackage{float}
\usepackage{fullpage}
\usepackage{graphicx}
\usepackage{hyperref}
\usepackage{latexsym}
\usepackage{listings}
\usepackage{pxfonts}
\usepackage{stackrel}
\usepackage{subfig}
\usepackage{tikz}
\usepackage{titlesec}

% Commands
\newcommand{\N}{\mathbb{N}}
\newcommand{\R}{\mathbb{R}}
\newcommand{\abs}[1]{\left\lvert#1\right\rvert}
\newcommand{\code}[2]{\lstinputlisting[caption={#2}]{#1}}
\newcommand{\margin}{\hspace{4pt}}
\newcommand{\norm}[1]{\left\lVert#1\right\rVert}

% Environments
\newenvironment{matlab}
	{\begin{figure}[hp]\centering\captionsetup{justification=centering}}
	{\end{figure}}

\newenvironment{rcases}
	{\left.\begin{aligned}}
	{\end{aligned}\right\rbrace}

% Python Syntax Highlighting
\definecolor{string_color}{RGB}{0, 161, 13}
\definecolor{comment_color}{RGB}{46, 46, 46}
\definecolor{keyword_color}{RGB}{0, 112, 191}
\definecolor{background_color}{RGB}{250, 250, 250}

\lstset{
    framesep=15pt,
    xleftmargin=15pt,
    xrightmargin=15pt,
    language=Python,
    captionpos=b,
    numbers=right,
    numberstyle=\small\ttfamily,
    frame=lines,
    showspaces=false,
    showtabs=false,
    breaklines=true,
    showstringspaces=false,
    breakatwhitespace=true,
    commentstyle=\color{comment_color}\textit,
    keywordstyle=\bfseries\color{keyword_color}\textbf,
    stringstyle=\color{string_color}\textit,
    morekeywords={self, lambda, __init__, __del__, __name__, for, in, not, and, or, :},
    basicstyle=\small\ttfamily,
    tabsize=4,
    keepspaces=true,
    columns=flexible,
    backgroundcolor=\color{background_color}
}

% Links
\hypersetup{
    colorlinks=true,
    linkcolor=blue,
    filecolor=magenta,
    urlcolor=cyan,
}

% Lengths
\setlength{\parindent}{0in}
\setlength{\oddsidemargin}{0in}
\setlength{\textwidth}{6.5in}
\setlength{\textheight}{10in}
\setlength{\topmargin}{-1.0in}
\setlength{\headheight}{18pt}

\titlespacing*{\subsection}
{0pt}{5.5ex plus 1ex minus .2ex}{4.3ex plus .2ex}

\title{\hugeΑλγοριθμική Επιχειρησιακή Έρευνα\\Πέμπτη Εργασία}
\author{Σιώρος Βασίλειος\\Ανδρινοπούλου Χριστίνα}
\date{Δεκέμβριος 2019}

\begin{document}

\maketitle

\pagenumbering{gobble}

\pagebreak


\subsection*{1. Consider the linear programming problem
	\begin{align*}
		\min \quad x_{1} - x_{2} \\
		\begin{aligned}
			\textup{s.t.}\quad
			2 \cdot x_{1} + 3 \cdot x_{2} - x_{3} + x_{4} & \leq 0 \\
			3 \cdot x_{1} + \cdot x_{2} + 4 \cdot x_{3} - 2 \cdot x_{4} & \geq 3 \\    
			- x_{1} - x_{2} + 2 \cdot x_{3} + x_{4} & = 6 \\
			x_{1} & \leq 0 \\
			x_{2}, x_{3} & \geq 0 
		\end{aligned}
	\end{align*}
	Write down the corresponding dual problem.}

Θέτω: \\ 

\begin{align*}
	c &= \begin{bmatrix}
			1 & -1 & 0 & 0
		\end{bmatrix} \\
	A &= \begin{bmatrix}
			2 & 3 & -1 & 1 \\
			3 & 1 & 4 & -2 \\
			-1 & -1 & 2 & 1
		\end{bmatrix} \\
	b &= \begin{bmatrix}
			0 & 3 & 6 & 0
		\end{bmatrix}
\end{align*}

Το δυικό πρόβλημα θα είναι: \\

\begin{align*}
	\max \quad 3 \cdot p_{2} + 6 \cdot p_{3} \\
	\begin{aligned}
		\textup{s.t.}\quad
		p_{1} & \leq 0 \\
		p_{2} & \geq 0 \\
		p_{3} & \quad free \\
		2 \cdot p_{1} + 3 \cdot p_{2} - p_{3} & \geq 1 \\
		3 \cdot p_{1} + p_{2} - p_{3} & \leq -1 \\
		-p_{1} + 4 \cdot p_{2} + 2 \cdot p_{3} & \leq 0 \\
		p_{1} - 2 \cdot p_{2} + p_{3} & = 0
	\end{aligned}
\end{align*}

\vspace{2in}

\pagebreak

\subsection*{2. Consider the primal problem
	\begin{align*}
		\min \quad c^{T} \cdot x \\
		\begin{aligned}
			\textup{s.t.}\quad
			A \cdot x & \geq b \\
			x & \geq 0 
		\end{aligned}
	\end{align*}
	Form the dual problem and convert it into an equivalent minimization problem. Derive a set of
	conditions on the matrix A and the vectors b,c under which the dual is identical to the primal.}

Το δυικό πρόβλημα είναι το εξής: \\

\begin{align*}
	\max \quad p^{T} \cdot b \\
	\begin{aligned}
		\textup{s.t.}\quad	
		p^{T} \cdot A & \leq c^{T} \\
		p & \geq 0 
	\end{aligned}
\end{align*}

To ισοδύναμο πρόβλημα ελαχιστοποίησης του δυικού προβλήματος που μολις αναφέραμε είναι: \\

\begin{align*}
	\min \quad -p^{T} \cdot b \\
	\begin{aligned}
		\textup{s.t.}\quad
		-p^{T} \cdot A & \geq -c^{T} \\
		p & \geq 0 
	\end{aligned}
\end{align*}

Για να μπορεί το παραπάνω πρόβλημα ελαχιστοποίησης να είναι ισοδύναμο με το αρχικό πρέπει να ισχύουν τα εξής: \\

\begin{align*}
	&\bullet \quad b = -c \\
	&\bullet \quad \text{ο πίνακας A πρέπει να είναι αντισυμμετρικός} 
\end{align*}

Να σημειώσουμε στο σημείο αυτό ότι αντισυμμετρικός είναι ένας τετραγωνικός πίνακας, ο οποίος είναι ίσος με τον αρνητικό του ανάστροφο, δηλαδή  \\

\begin{align*}
	A = -A^{T}
\end{align*}

Αν ισχύουν τα παρπάνω, ισχύει και ότι: \\

\begin{align*}
	-p^{T} \cdot A \geq -c^{T} &\Leftrightarrow \\
	-A^{T} \cdot p \geq -c &\Leftrightarrow \\
	A \cdot p \geq b
\end{align*}

και επίσης ισχύει και ότι: \\

\begin{align*}
	-p^{T} \cdot b &= \\ 
	-p^{T} \cdot \left( - c \right) &= \\
	p^{T} \cdot c &= \\
	c^{T} \cdot p
\end{align*}


Συνεπώς το ισοδύναμο πρόβλημα ελαχιστοποίησης του δυικού ταυτίζεται με το αρχικό πρόβλημα αν ισχύουν οι παραπάνω δύο ειδικές συνθήκες.

\vspace{2in}

\pagebreak

\subsection*{3. The purpose of this exercise is to show that solving linear programming
problems is no harder than solving systems of linear inequalities. Suppose that we are given
a subroutine which, given a system of linear inequalities either produces a solution or decides
that no solution exists. Construct a simple algorithm that uses a single call to this subroutine
and which finds an optimal solution to any linear programming problem that has an optimal
solution.}

Έστω το εξής πρόβλημα γραμμικού προγραμματισμού

\begin{align*}
    \max \quad c \cdot x \\
    \begin{aligned}
   		\textup{s.t.}\quad
    	  A \cdot  x & \leq b \\
	      x & \geq 0
    \end{aligned}
\end{align*}

και το δϋικό του

\begin{align*}
    \min \quad y \cdot  b \\
    \begin{aligned}
   		\textup{s.t.}\quad
 		A^T \cdot  y & \leq c \\
		y & \geq 0
    \end{aligned}
\end{align*}

Ορίζουμε το παρακάτω σύστημα γραμμικών ανισώσεων

\begin{align*}
    A \cdot & x \leq b \\
    & x \geq 0 \\
    A^T \cdot & y \leq c \\
    & y \geq 0 \\
    c \cdot x & \geq y \cdot b
\end{align*}

\begin{itemize}
    \item Oι 4 πρώτες ανισώσεις εξασφαλίζουν, όπως είναι προφανές, την εφικτότητα των λύσεων.
    \item Η τελευταία ανίσωση εξασφαλίζει τη βελτιστότητα των λύσεων. \\

    Από το Θεώρημα Ασθενούς Δϋικότητας γνωρίζουμε ότι αν τα \( x \) και \( y \)
    αποτελούν δυνατές λύσεις του πρωτεύοντος και του δϋικού προβλήματος αντίστοιχα
    τότε ισχύει \( c \cdot x \leq y \cdot b \). \\

    Επομένως η ανίσωση \( c \cdot x \geq y \cdot b \) ισχύει αποκλειστικά στην περίπτωση
    \( c \cdot x = y \cdot b \), γεγονός που συνεπάγεται, λόγω
    του Θεώρημα Ισχυρής Δϋικότητας, ότι τα \( x \) και \( y \)
    αποτελούν βέλτιστες λύσεις του πρωτεύοντος και του δϋικού προβλήματος αντίστοιχα.
\end{itemize}

\pagebreak

Έτσι, δοθέντος οποιουδήποτε προβλήματος γραμμικού προγραμματισμού αρκεί να ορίσουμε το
αντίστοιχο σύστημα γραμμικών ανισώσεων, όπως παραπάνω, και να καλέσουμε την υπορουτίνα μας επί αυτού. \\

Αν η υπορουτίνα αποφανθεί ότι δεν υπάρχει λύση στο σύστημα γραμμικών ανισώσεων,
αυτό συνεπάγεται ότι είτε δεν υπάρχει εφικτή λύση για το δοθέν γραμμικό προβλήμα
ή ότι δεν υπάρχει βέλτιση λύση. \\

Διαφορετικά επιστρέφονται οι βέλτιστες λύσεις τόσο του πρωτεύοντος όσο και του δϋικού προβλήματος. \\

\vspace{2in}

\pagebreak

\subsection*{4. Let A be a symmetric matrix. Consider the linear program
	\begin{align*}
		\min \quad c^{T} \cdot x \\
		\begin{aligned}
	   		\textup{s.t.}\quad
			A \cdot x & \geq c \\
			x & \geq 0 
		\end{aligned}
	\end{align*}
	Prove that if \({x*}\) satisfies \(A \cdot {x*} = c\) and \({x*} \geq 0\) then \({x*}\) is an optimal solution.}

Αν υποθέσουμε ότι \({x*}\) ικανοποιεί το \(A \cdot {x*} = c\) και ότι \({x*} \geq 0\), πρέπει να αποδείξουμε ότι το \({x*}\) είναι βέλτιστη λύση. \\

Ο πίνακας \(A\) είναι συμμετρικός (άρα και τετραγωνικός) και συνεπώς ισχύει ότι είναι ίσος με τον ανάστροφό του\\

\begin{align*}
	A = A^{T}
\end{align*}

Σχηματίζουμε το δυικό του παραπάνω προβλήματος γραμμικού προγραμματισμού ως εξής \\

\begin{align*}
	\max \quad p^{T} \cdot x \\
	\begin{aligned}
		\textup{s.t.}\quad
		p^{T} \cdot A & \leq c \\
		p & \geq 0 
	\end{aligned}
\end{align*}

Αν η βέλτιστη λύση του δυικού \({p*}\) είναι ίδια με τη βέλτιστη λύση του αρχικού προβλήματος, \({x*}\), τότε ισχυεί ότι \\

\begin{align*}
	{p*} = {x*} \geq 0
\end{align*}

Θα ισχύει ότι \\

\begin{align*}
	{p*}^{T} \cdot A &= A^{T} \cdot {p*} \Leftrightarrow \\
	{p*}^{T} \cdot A &= A \cdot {p*} \Leftrightarrow \\
	{p*}^{T} \cdot A &= A \cdot {x*} \Leftrightarrow \\
	{p*}^{T} \cdot A &= c
\end{align*}

Επίσης, ισχύει ότι \\

\begin{align*}
	c^{T} \cdot {x*} &= {x*}^{T} \cdot c \Leftrightarrow \\
	c^{T} \cdot {x*} &= {p*}^{T} \cdot c 	
\end{align*}

Ξέρουμε ότι αν \({x*}\) και \({p*}\) είναι εφικτές λύσεις στο αρχικό πρόβλημα γραμμικού προγραμματισμού και στο δυικό του αντίστοιχα και \({p*}^{T} \cdot b = c^{T} \cdot {x*}\), τότε τα \({x*}\) και τα \({p*}\) είναι οι βέλτιστες λύσεις. \\

Συνεπώς, η \({x*}\) είναι βέλτιστη λύση για το πρόβλημα γραμμικού προγραμματισμού.

\vspace{2in}

\pagebreak

\subsection*{5. Write down the proof of the \textit{Complimentary Slackness Theorem}.}

Έστω το πρωτεύον πρόβλημα \\

\begin{align*}
    \max \quad c \cdot x \\
    \begin{aligned}
    	\textup{s.t.}\quad
 		A \cdot x & \leq b \\
		x & \geq 0
    \end{aligned}
\end{align*}

και το δϋικό του

\begin{align*}
    \min \quad b \cdot y \\
    \begin{aligned}
    	\textup{s.t.}\quad
   		A^T \cdot y & \geq c \\
		y & \geq 0
    \end{aligned}
\end{align*}

Έστω \( x \) και \( y \) εφικτές λύσεις στο πρωτεύον και στο δϋικό πρόβλημα αντίστοιχα. \\

Το \( x \) και το \( y \) αποτελούν βέλτιστες λύσεις του πρωτεύοντος και του δϋικού
αν και μόνο αν ισχύει ότι: \\

\begin{itemize}
    \item \( (b_i - \sum_{j=1}^{n} {a_{ij}} \cdot x_j) \cdot y_i = 0 \margin \forall i \in \{ 1, \ldots, m\} \)
    \item \( (\sum_{i = 1}^{n} a_{ji} \cdot y_i - c_j) \cdot x_j = 0 \margin \forall j \in \{ 1, \ldots, n\} \)
\end{itemize}

\[ \underline{\textbf{ΑΠΟΔΕΙΞΗ}} \]

Δεδομένου ότι οι λύσεις \( x \) και \( y \) είναι εφικτές έχουμε:

\begin{align*}
    \begin{rcases}
        A^T & \cdot y \geq c \\
        x & \geq 0
    \end{rcases}
    \Rightarrow \\
    x^T \cdot A^T \cdot y \geq x^T \cdot c = c \cdot x
\end{align*}

\begin{align*}
    \begin{rcases}
        A & \cdot x \leq b \\
        y & \geq 0
    \end{rcases}
    \Rightarrow \\
    y^T \cdot A \cdot x \leq y^T \cdot b = b \cdot y
\end{align*}

Από τα παραπάνω, δεδομένου ότι \( x^T \cdot A^T \cdot y \) είναι ένας \( 1 \times 1 \)
πίνακας και ως εκ τούτου \( x^T \cdot A^T \cdot y = (x^T \cdot A^T \cdot y)^T = y^T \cdot A \cdot x\)
προκύπτει

\[ c \cdot x = x^T \cdot c \leq x^T \cdot A^T \cdot y = y^T \cdot A \cdot x \leq y^T \cdot b = b \cdot y \]

Βάσει Ισχυρής Δϋικότητας, τα \( x \) και \( y \) αποτελούν βέλτιστες λύσεις στα αντίστοιχα
γραμμικά προβλήματα αν και μόνο αν \( c \cdot x = b \cdot y \) και άρα αν και μόνο αν

\begin{itemize}
    \item \( x^T \cdot c = x^T \cdot A^T \cdot y \Leftrightarrow x \cdot (A^T \cdot y - c) = 0 \)
    \item \( y^T \cdot A \cdot x = y^T \cdot b \Leftrightarrow y \cdot (b - A \cdot x) = 0 \)
\end{itemize}

Σημειώνουμε ότι

\[ 0 = x \cdot (A^T \cdot y - c)  = \sum_{j = 1}^{n}(x_j \cdot (\sum_{i = 1}^{m} a_{ji} \cdot y_i - c_j)) \]

και δεδομένου ότι \( x_j \geq 0 \) και \( \sum_{i = 1}^{m} a_{ji} \cdot y_i - c_j \geq 0 \)
\( \forall j \in \{1, \ldots, n \} \) καταλήγουμε στο συμπέρασμα ότι

\[ x_j \cdot (\sum_{i = 1}^{m} a_{ji} \cdot y_i - c_j) \geq 0 \forall j \in \{ 1, \ldots, n \} \]

Από τα παραπάνω προκύπτει ότι κάθε όρος του παραπάνω αθροίσματος είναι υποχρεωτικά ίσος με 0
που αντιστοιχεί στην πρώτη προϋπόθεση του Θεωρήματος. \\

Ομοίως αποδεικνύεται ότι

\[ y \cdot ( b - A \cdot x ) = 0 \Leftrightarrow y_i \cdot ( b_i - \sum_{j = 1}^{n} a_{ij} \cdot x_j ) = 0 \margin \forall i \in \{ 1, \ldots, m \} \]

\vspace{2in}

\pagebreak

\end{document}
