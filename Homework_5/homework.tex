
\documentclass[12pt]{article}

\usepackage[utf8]{inputenc}
\usepackage[greek, english]{babel}

% Packages
\usepackage{alphabeta}
\usepackage{amsmath}
\usepackage{amsthm}
\usepackage{caption}
\usepackage{color}
\usepackage{float}
\usepackage{fullpage}
\usepackage{graphicx}
\usepackage{hyperref}
\usepackage{latexsym}
\usepackage{listings}
\usepackage{pxfonts}
\usepackage{stackrel}
\usepackage{subfig}
\usepackage{tikz}
\usepackage{titlesec}

% Commands
\newcommand{\N}{\mathbb{N}}
\newcommand{\R}{\mathbb{R}}
\newcommand{\abs}[1]{\left\lvert#1\right\rvert}
\newcommand{\code}[2]{\lstinputlisting[caption={#2}]{#1}}
\newcommand{\margin}{\hspace{4pt}}
\newcommand{\norm}[1]{\left\lVert#1\right\rVert}

% Environments
\newenvironment{matlab}
	{\begin{figure}[hp]\centering\captionsetup{justification=centering}}
	{\end{figure}}

\newenvironment{rcases}
	{\left.\begin{aligned}}
	{\end{aligned}\right\rbrace}

% Python Syntax Highlighting
\definecolor{string_color}{RGB}{0, 161, 13}
\definecolor{comment_color}{RGB}{46, 46, 46}
\definecolor{keyword_color}{RGB}{0, 112, 191}
\definecolor{background_color}{RGB}{250, 250, 250}

\lstset{
    framesep=15pt,
    xleftmargin=15pt,
    xrightmargin=15pt,
    language=Python,
    captionpos=b,
    numbers=right,
    numberstyle=\small\ttfamily,
    frame=lines,
    showspaces=false,
    showtabs=false,
    breaklines=true,
    showstringspaces=false,
    breakatwhitespace=true,
    commentstyle=\color{comment_color}\textit,
    keywordstyle=\bfseries\color{keyword_color}\textbf,
    stringstyle=\color{string_color}\textit,
    morekeywords={self, lambda, __init__, __del__, __name__, for, in, not, and, or, :},
    basicstyle=\small\ttfamily,
    tabsize=4,
    keepspaces=true,
    columns=flexible,
    backgroundcolor=\color{background_color}
}

% Links
\hypersetup{
    colorlinks=true,
    linkcolor=blue,
    filecolor=magenta,
    urlcolor=cyan,
}

% Lengths
\setlength{\parindent}{0in}
\setlength{\oddsidemargin}{0in}
\setlength{\textwidth}{6.5in}
\setlength{\textheight}{10in}
\setlength{\topmargin}{-1.0in}
\setlength{\headheight}{18pt}

\titlespacing*{\subsection}
{0pt}{5.5ex plus 1ex minus .2ex}{4.3ex plus .2ex}

\title{\hugeΑλγοριθμική Επιχειρησιακή Έρευνα\\Πέμπτη Εργασία}
\author{Σιώρος Βασίλειος\\Ανδρινοπούλου Χριστίνα}
\date{Δεκέμβριος 2019}

\begin{document}

\maketitle

\pagenumbering{gobble}

\pagebreak


\subsection*{1. Consider the linear programming problem
min x1 \ensuremath{-} x2
s.t. 2x1 + 3x2 \ensuremath{-} x3 + x4 \ensuremath{\leq} 0
3x1 + x2 + 4x3 \ensuremath{-} 2x4 \ensuremath{\geq} 3
\ensuremath{-} x1 \ensuremath{-} x2 + 2x3 + x4 = 6
x \ensuremath{\leq} 0
x2, x3 \ensuremath{\geq} 0
Write down the corresponding dual problem.}

\vspace{2in}

\pagebreak

\subsection*{2. Consider the primal problem
min c'x
s.t. Ax \ensuremath{\geq} b
x \ensuremath{\geq} 0
Form the dual problem and convert it into an equivalent minimization problem. Derive a set of
conditions on the matrix A and the vectors b,c under which the dual is identical to the primal.}

\vspace{2in}

\pagebreak

\subsection*{3. The purpose of this exercise is to show that solving linear programming
problems is no harder than solving systems of linear inequalities. Suppose that we are given
a subroutine which, given a system of linear inequalities either produces a solution or decides
that no solution exists. Construct a simple algorithm that uses a single call to this subroutine
and which finds an optimal solution to any linear programming problem that has an optimal
solution.}

\vspace{2in}

\pagebreak

\subsection*{4. Let A be a symmetric matrix. Consider the linear program
min c'x
s.t. Ax \ensuremath{\geq} c
x \ensuremath{\geq} 0
Prove that if x\ensuremath{*} satisfies Ax\ensuremath{*} = c and x\ensuremath{*} \ensuremath{\geq} 0 then x\ensuremath{*} is an optimal solution.}

\vspace{2in}

\pagebreak

\subsection*{5. Write down the proof of the \textit{Complimentary Slackness Theorem}.}

Έστω το πρωτεύον πρόβλημα \\

\begin{align*}
    \max \margin c & \cdot x \\
    A \times x & \leq b \\
    x & \geq 0
\end{align*}

και το δϋικό του

\begin{align*}
    \min \margin b & \cdot y \\
    A^T \times y & \geq c \\
    y & \geq 0
\end{align*}

Έστω \( x \) και \( y \) εφικτές λύσεις στο πρωτεύον και στο δϋικό πρόβλημα αντίστοιχα. \\

Το \( x \) και το \( y \) αποτελούν βέλτιστες λύσεις του πρωτεύοντος και του δϋικού
αν και μόνο αν ισχύει ότι: \\

\begin{itemize}
    \item \( (b_i - \sum{j = 1}{n} a_{ij} \cdot x_j) \cdot y_i = 0 \margin \forall i \in \{ 1, \ldots, m\} \)
    \item \( (\sum{i = 1}{n} a_{ji} \cdot y_i - c_j) \cdot x_j = 0 \margin \forall j \in \{ 1, \ldots, n\} \)
\end{itemize}

\[ \underline{\textbf{ΑΠΟΔΕΙΞΗ}} \]

Δεδομένου ότι οι λύσεις \( x \) και \( y \) είναι εφικτές έχουμε:

\begin{align*}
    \begin{rcases}
        A^T & \times y \geq c
        x & \geq 0
    \end{rcases}
    \Rightarrow \\
    x^T \times A^T \times y \geq x^T \times c = c \cdot x
\end{align*}

\begin{align*}
    \begin{rcases}
        A & \times x \leq b \\
        y & \geq 0
    \end{rcases}
    \Rightarrow \\
    y^T \times A \times x \leq y^T \times b = b \cdot y
\end{align*}

Από τα παραπάνω, δεδομένου ότι \( x^T \times A^T \times y \) είναι ένας \( 1 \times 1 \)
πίνακας και ως εκ τούτου \( x^T \times A^T \times y = (x^T \times A^T \times y)^T = y^T \times A \times x\)
προκύπτει

\[ c \cdot x = x^T \times c \leq x^T \times A^T \times y = y^T \times A \times x \leq y^T \times b = b \cdot y \]

Βάσει Ισχυρής Δϋικότητας, τα \( x \) και \( y \) αποτελούν βέλτιστες λύσεις στα αντίστοιχα
γραμμικά προβλήματα αν και μόνο αν \( c \dot x = b \cdot y \) και άρα αν και μόνο αν

\begin{itemize}
    \item \( x^T \times c = x^T \times A^T \times y \Leftrightarrow x \cdot (A^T \times y - c) = 0 \)
    \item \( y^T \times A \times x = y^T \times b \Leftrightarrow y \cdot (b - A \times x) = 0 \)
\end{itemize}

Σημειώνουμε ότι

\[ 0 = x \cdot (A^T \times y - c)  = \sum_{j = 1}^{n}(x_j \cdot (\sum_{i = 1}^{m} a_{ji} \cdot y_i - c_j)) \]

και δεδομένου ότι \( x_j \geq 0 \) και \( \sum_{i = 1}^{m} a_{ji} \cdot y_i - c_j \geq 0 \)
\( \forall j \in \{1, \ldots, n \} \) καταλήουμε στο συμπέρασμα ότι

\[ x_j \cdot (\sum_{i = 1}^{m} a_{ji} \cdot y_i - c_j) \geq 0 \forall j \in \{ 1, \ldots, n \} \]

Από τα παραπάνω προκύπτει ότι κάθε όρος του παραπάνω αθροίσματος είναι υποχρεωτικά ίσος με 0
που αντιστοιχεί στην πρώτη προϋπόθεση του Θεωρήματος. \\

Ομοίως αποδεικνύεται ότι

\[ y \cdot ( b - A \times x ) = 0 \Leftrightarrow y_i \cdot ( b_i - \sum_{j = 1}^{n} a_{ij} \cdot x_j ) = 0 \margin \forall i \in \{ 1, \ldots, m \} \]

\vspace{2in}

\pagebreak

\end{document}
