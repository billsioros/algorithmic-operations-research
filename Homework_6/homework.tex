
\documentclass[12pt]{article}

\usepackage[utf8]{inputenc}
\usepackage[greek, english]{babel}

% Packages
\usepackage{MnSymbol}
\usepackage{alphabeta}
\usepackage{amsfonts}
\usepackage{amsmath}
\usepackage{amsthm}
\usepackage{caption}
\usepackage{graphicx}
\usepackage{latexsym}
\usepackage{stackrel}
\usepackage{titlesec}

% Commands
\newcommand{\R}{\mathbb{R}}
\newcommand{\N}{\mathbb{N}}
\newcommand{\norm}[1]{\left\lVert#1\right\rVert}
\newcommand{\margin}{\hspace{4pt}}
\newcommand{\centered}[1]{\begin{align*}#1\end{align*}}
\newcommand{\plot}{\includegraphics}

% Environments
\newenvironment{rcases}
	{\left.\begin{aligned}}
	{\end{aligned}\right\rbrace}

\newenvironment{matlab}
	{\begin{figure}[hp]\centering\captionsetup{justification=centering}}
	{\end{figure}}

\setlength{\parindent}{0in}
\setlength{\oddsidemargin}{0in}
\setlength{\textwidth}{6.5in}
\setlength{\textheight}{10in}
\setlength{\topmargin}{-1.0in}
\setlength{\headheight}{18pt}

\titlespacing*{\subsection}
{0pt}{5.5ex plus 1ex minus .2ex}{4.3ex plus .2ex}

\title{\hugeΑλγοριθμική Επιχειρησιακή Έρευνα\\Έκτη Εργασία}
\author{Σιώρος Βασίλειος\\Ανδρινοπούλου Χριστίνα}
\date{Ιανουάριος 2020}

\begin{document}

\maketitle

\pagenumbering{gobble}

\pagebreak


\subsection*{1. Define the 0-1 knapsack problem.}



\vspace{2in}

\pagebreak

\subsection*{2. Give a dynamic programming solution to the 0-1 knapsack problem.}


\vspace{2in}

\pagebreak

\subsection*{3. Give real world applications of knapsack problem.}

Το knapsack problem (πρόβλημα του σακιδίου) έχει πολλές πρακτικές εφαρμογές. Συγκεκριμένα, εφαρμόζεται άμεσα  στα προβλήματα που συναντάμε στη ζωή, που σχετίζονται με "πηγές" που έχουν συγκεκριμένη "αξία" και επιθυμούμε να επιλέξουμε από αυτές, περιοριζοντας όσο το δυνατό καλύτερα το τελικό κόστος της επιλογής μας.\\

Στη συνέχεια θα αναφέρουμε μερικά τέτοια παραδείγματα: \\
\(\bullet\) Ως πρώτο παράδειγμα θα αναφέρουμε εκείνο, στο οποίο οφείλει και το όνομά του το knapsack problem. Η προετοιμασία της βαλίτσας για κάποιο ταξίδι. Η βαλίτσα έχει συγκεκριμένο διαθέσιμο χώρο για να φιλοξενήσει τα πράγματά μας. Πρέπει να γίνει η καλύτερη επιλογή αντικειμένων (ρούχα, προσωπικά αντικείμενα κτλ.), ώστε να υπάρχουν όλα τα απαραίτητα μέσα στη βαλίτσα (κι αν ταξιδεύουμε με αεροπλάνο, το βάρος όλων των αντικειμένων να μην υπερβαίνει κάποιο ορισμένο όριο κιλών). \\

\begin{figure}
	\centering
	\includegraphics[scale=0.45]{./knapsack-example.png}
	\caption{Μόνο με χιουμοριστική διάθεση..}
\end{figure}

\(\bullet\) Η αποθήκευση εμπορευμάτων σε αποθήκες. Ο χώρος της αποθήκης είναι καθορισμένος και δεν μεταβάλλεται και πρέπει σε αυτόν το χώρο να μπορέσουν να χωρέσουν όσο το δυνατό περισσότερα εμπορεύματα. Απλουστεύοντας τα δεδομένα, μπορόυμε να υποθέσουμε ότι περισσότερα προϊόντα στην αποθήκη, σημαίνει αύξηση των κερδών της επιχείρησης. \\

\(\bullet\) Οι κατάλληλες επενδύσεις. Έστω  ότι επιθυμούμε να επενδύσουμε όλα ή ένα  μέρος  από τα κεφάλαια. Έχουμε να επιλέξουμε ανάμεσα  σε κάποιες πιθανές επενδύσεις. Κάθε επένδυση έχει ένα συγκεκριμένο κόστος και ένα αναμενόμενο κέρδος. Η βέλτιστη επιλογή μπορεί να βρεθεί με το πρόβλημα του σακιδίου. \\ 

\(\bullet\) Η επιλογή υπαλλήλων για μία επιχείρηση. Υπάρχουν πολλοί εν δυνάμει υπάλληλοι, με διαφορετικές ικανότητες, ταλέντα, γνώσεις, πτυχία και πιστοποιήσεις, αλλά ταυτόχρονα και με διαφορετικές μισθολογικές απαιτήσεις. Η επιχείρηση είναι σε θέση να διαθέτει ένας συγκεκριμένο ποσό για μισθοδοσίες και μπορεί να παρέχει καθορισμένες παροχές στους εργαζομένους της (π.χ. κτηριακές υποδομές). Ποιά είναι η καλύτερη επιλογή εργαζομένων, ώστε η επιχείρηση να αυξήσει τα κέρδη της; \\

\(\bullet\) Στο σημείο αυτό θα αναφερθούμε σε ένα πραγματικό παράδειγμα εφαρμογής του προβλήματος του σακιδίου. Η εταιρεία Trashy Bags είναι μία εταιρεία ευαισθητοποιημένη στο θέμα των πλαστικών απορριμάτων, που είναι ένα θέμα ζωτικής σημασίας για τον πλανήτη. Επιλέγει να ανακυκλώνει και να επαναχρησιμοποιεί το πλαστικό. Μία ομάδα συλλεκτών συλλέγει από τους δρόμους γύρω στα 15 εκατομμύρια πλαστικές σακούλες. Τελικά, από το πλαστικό που έχει συλλέξει κατασκευάζει τσάντες και αδιάβροχα. Άρα, στην περίπτωση αυτή, η εταιρεία επιθυμεί να αυξήση την παραγωγή της, επιλέγοντας τα καλύτερα (πλαστικά) αντικείμενα, με σκοπό να παραχθεί ο απαιτούμενος  αριθμός  παραγόμενων προϊόντων και τελικά η εταιρεία να αυξήσει τα κέρδη της. Ο περιορισμός εδώ σχετίζεται με το κόστος που προκύπτει από το κόστος συλλογής απορριμμάτων από πλαστικές σακούλες,  το  κόστος  επεξεργασίας  και  το  κόστος  πώλησης  των προϊόντων. \\

\(\bullet\) Η επιλογή διαφημίσεων σε κάποιο ραδιοφωνικό/τηλεοπτικό σταθμό. Υπάρχει συγκεκριμένο πλήθος διαφημίσεων. Κάθε διαφήμιση έχει μία τιμή και κάποιον χρόνο. Στόχος είναι να μεταδοθούν όσο το δυνατό περισσότερες κερδοφόρες διαφημίσεις, χωρίς όμως να ξεπεραστεί ένα προκαθορισμένο χρονικό όριο. \\

\vspace{2in}

\pagebreak

\subsection*{4. Define the Subset sum problem and give a dynamic programming solution for
	it. Write down the difference between the subset sum and the knapsack problem.}



\vspace{2in}

\pagebreak

\end{document}
