\documentclass[xcolor=dvipsnames, 11pt]{beamer}
\usetheme{Warsaw}

\usepackage{ucs}
\usepackage[utf8x]{inputenc}
\usepackage[greek,english]{babel}
\newcommand{\en}{\selectlanguage{english}}
\newcommand{\el}{\selectlanguage{greek}}

\setbeamertemplate{itemize items}[ball]
\setbeamertemplate{itemize subitem}[ball]

\begin{document}

	\title{An Integer Programming Model for the Sudoku Problem}
	\subtitle{\el Παρουσίαση του \en paper \el και εμβάθυνση}
	
	\author[Bartlett, Chartier, Langville, Rankin] % (optional, for multiple authors)
	{Andrew C. Bartlett\inst{1} 
		\and Timothy P. Chartier\inst{2} 
		\and Amy N. Langville\inst{3} 
		\and Timothy D. Rankin\inst{4}}
	\institute{\inst{1} Department of Mathematics, College of Charleston, Charleston, SC, USA \and
			\inst{2} Department of Mathematics, Davidson College, Davidson, NC, USA \and
			\inst{3} Department of Mathematics, College of Charleston, Charleston, SC, USA \and
			\inst{4} Department of Mathematics, Davidson College, Davidson, NC, USA}
	\date{May 3, 2008}
	
	
	\frame{\titlepage}
	
	\begin{frame}
		\frametitle{\el Γενικά}
 
		\begin{itemize}
			\item \el Το \en Sudoku \el είναι ένα πάζλ βασισμένο στη λογική.
			\item Στόχος του παιχνιδιού είναι ο παίκτης να συμπληρώσει τα κενά κελιά ενός ημιτελώς συμπληρωμένου πίνακα μεγέθους \(n \times n\) με τους κατάλληλους ακεραίους που ανήκουν στο διάστημα \(\left[1,\dots,n \right]\), με τέτοιον τρόπο ώστε κάθε γραμμή, κάθε στήλη και κάθε υποπίνακας μεγέθουν \(m \times m\) να περιέχει όλους τους ακεραίους του διαστήματος \(\left[1,\dots,n \right]\) ακριβώς μία φορά τον καθέναν. 
		\end{itemize}    
	\end{frame}

	\begin{frame}
		\frametitle{ \el Ιστορικά στοιχεία}
		\begin{itemize}
			\item \el Δημιουργός του εν λόγω παιχνιδιού υπήρξε ο Αμερικανός αρχιτέκτονας \en Howard Garns \el (Μάρτιος 1905 - Οκτώβριος 1989) το 1979. 
			\item Το παιχνίδι αρχικά ονομάστηκε \en “Number Place” \el και δημοσιεύτηκε στο περιοδικό \en Dell Pencil Puzzles \& Word Games.
			\item \el Ένα χρόνο μετά το παιχνίδι έγινε ιδιαίτερα δημοφιλές στην Ιαπωνία και μετονομάστηκε σε \en “suji wa dokushin ni kagiru”(SUDOKU).
		\end{itemize}
		\begin{figure}
			\includegraphics[scale=0.1]{./Figures/Howard Garns.jpeg}
			\caption{Howard Garns}
		\end{figure}
	\end{frame}
	
	\begin{frame}
		\frametitle{ \el Μελέτη}
		\el Στην παρούσα εργασία θα μελετήσουμε σε βάθος:
		\begin{itemize}
			\item \el Μαθηματικές μεθοδολογίες για την επίλυση \en sudoku.
				\begin{itemize}
					\item \el Μοντελοποίηση του προβλήματος της επίλυσης του παζλ ως πρόβλημα γραμμικού προγραμματισμού.
					\item Επαλήθευση με χρήση \en MATLAB .
					\item \el Παραλλαγές του κλασσικού \en sudoku.
				\end{itemize}
			\item \el Μαθηματικές τεχνικές για τη δημιουργία \en sudoku.
			\begin{itemize}
				\item \el Δημιουργία \en sudoku \el με \en bruteforce.
				\item \el Δημιουργία \en sudoku \el από παλαιότερο παζλ.
			\end{itemize}
		\end{itemize}
	\end{frame}

	\begin{frame}
	\frametitle{References \el για εμβάθυνση}
	
	\begin{itemize}
		\item J. F. Crook, A Pencil-and-Paper Algorithm for Solving Sudoku Puzzles, Notices of the AMS (April 2009)
		\item Andrew C. Stuart, Sudoku Creation and Grading (February 2007 - updated January 2012)
		\item Arnab Kumar Maji, Sunanda Jana, Sudipta Roy, Rajat Kumar Pal, An Exhaustive Study on different Sudoku Solving Techniques, International Journal of Computer Science Issues, Vol. 11, Issue 2, No 1, March 2014
		\item Radek Pelánek, Human Problem Solving:Sudoku Case Study (January 2011)
		\item Rohit Iyer, Amrish Jhaveri, Krutika Parab, A Review of Sudoku Solving using Patterns, International Journal of Scientific and Research Publications, Volume 3, Issue 5, May 2013
		\item Rhyd Lewis, Metaheuristics can Solve Sudoku Puzzles
	\end{itemize}    
\end{frame}

\end{document}