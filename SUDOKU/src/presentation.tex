
\documentclass[12pt]{article}

\usepackage[utf8]{inputenc}
\usepackage[greek, english]{babel}

% Packages
\usepackage{alphabeta}
\usepackage{amsmath}
\usepackage{amsthm}
\usepackage{caption}
\usepackage{color}
\usepackage{float}
\usepackage{fullpage}
\usepackage{graphicx}
\usepackage{hyperref}
\usepackage{latexsym}
\usepackage{listings}
\usepackage{pxfonts}
\usepackage{stackrel}
\usepackage{subfig}
\usepackage{tikz}
\usepackage{titlesec}

% Commands
\newcommand{\N}{\mathbb{N}}
\newcommand{\R}{\mathbb{R}}
\newcommand{\abs}[1]{\left\lvert#1\right\rvert}
\newcommand{\code}[2]{\lstinputlisting[caption={#2}]{#1}}
\newcommand{\margin}{\hspace{4pt}}
\newcommand{\norm}[1]{\left\lVert#1\right\rVert}

% Environments
\newenvironment{matlab}
	{\begin{figure}[hp]\centering\captionsetup{justification=centering}}
	{\end{figure}}

\newenvironment{rcases}
	{\left.\begin{aligned}}
	{\end{aligned}\right\rbrace}

% Python Syntax Highlighting
\definecolor{string_color}{RGB}{0, 161, 13}
\definecolor{comment_color}{RGB}{46, 46, 46}
\definecolor{keyword_color}{RGB}{0, 112, 191}
\definecolor{background_color}{RGB}{250, 250, 250}

\lstset{
    framesep=15pt,
    xleftmargin=15pt,
    xrightmargin=15pt,
    language=Python,
    captionpos=b,
    numbers=right,
    numberstyle=\small\ttfamily,
    frame=lines,
    showspaces=false,
    showtabs=false,
    breaklines=true,
    showstringspaces=false,
    breakatwhitespace=true,
    commentstyle=\color{comment_color}\textit,
    keywordstyle=\bfseries\color{keyword_color}\textbf,
    stringstyle=\color{string_color}\textit,
    morekeywords={self, lambda, __init__, __del__, __name__, for, in, not, and, or, :},
    basicstyle=\small\ttfamily,
    tabsize=4,
    keepspaces=true,
    columns=flexible,
    backgroundcolor=\color{background_color}
}

% Links
\hypersetup{
    colorlinks=true,
    linkcolor=blue,
    filecolor=magenta,
    urlcolor=cyan,
}

% Lengths
\setlength{\parindent}{0in}
\setlength{\oddsidemargin}{0in}
\setlength{\textwidth}{6.5in}
\setlength{\textheight}{10in}
\setlength{\topmargin}{-1.0in}
\setlength{\headheight}{18pt}

\titlespacing*{\subsection}
{0pt}{5.5ex plus 1ex minus .2ex}{4.3ex plus .2ex}

\title{\hugeΥλοποίηση σε Python}
\author{Σιώρος Βασίλειος\\Ανδρινοπούλου Χριστίνα}
\date{Ιανουάριος 2020}

\begin{document}

\maketitle

\pagenumbering{gobble}

\pagebreak

\section{Modeling Sudoku Variations in Python}

The implementation was based on the paper \href{https://www.researchgate.net/publication/228615106_An_integer_programming_model_for_the_sudoku_problem}{\textit{"An Integer Programming Model
for the Sudoku Problem"}}. \\

We are going to be using the \href{https://pythonhosted.org/PuLP/}{\textbf{PuLP}} python library,
in order to model the given problems in terms of Linear Programming. \\

\subsection{The Classic Sudoku Solver class}

\begin{lstlisting}[caption={Initializing the \textbf{LpProblem} super class as a minimization problem}]
    self.matrix = matrix
    self.n = len(matrix)
    self.m = int(sqrt(self.n))

    super().__init__(
        name=f"{type(self).__name__}_solver_{self.n}_x_{self.n}".lower(),
        sense=LpMinimize
    )
\end{lstlisting}

\begin{lstlisting}[caption={Declaring our variables}]
    self.x = [
        [
            [
                LpVariable(
                    f"x_{i + 1:02d}_{j + 1:02d}_{k + 1:02d}", cat=LpBinary)
                for k in range(self.n)
            ] for j in range(self.n)
        ] for i in range(self.n)
    ]
\end{lstlisting}

\begin{lstlisting}[caption={Declaring the objective function}]
    self += 0
\end{lstlisting}

\pagebreak

\begin{lstlisting}[caption={Declaring that there should only be one \textit{k} in each column}]
    for j in range(self.n):
        for k in range(self.n):
            self += lpSum([self.x[i][j][k]
                            for i in range(self.n)]) == 1, f"in column {j + 1:02d} only one {k + 1:02d}"
\end{lstlisting}

\begin{lstlisting}[caption={Declaring that there should only be one \textit{k} in each row}]
    for i in range(self.n):
        for k in range(self.n):
            self += lpSum([self.x[i][j][k]
                            for j in range(self.n)]) == 1, f"in row {i + 1:02d} only one {k + 1:02d}"
\end{lstlisting}

\begin{lstlisting}[caption={Declaring that there should only be one \textit{k} in each submatrix}]
    for k in range(self.n):
        for p in range(self.m):
            for q in range(self.m):
                self += lpSum([
                    [
                        lpSum([
                            self.x[i][j][k]
                            for i in range(self.m * p, self.m * (p + 1))
                        ])
                    ]
                    for j in range(self.m * q, self.m * (q + 1))
                ]) == 1, f"in submatrix {p + 1:02d} {q + 1:02d} only one {k + 1:02d}"
\end{lstlisting}

\begin{lstlisting}[caption={Declaring that there should only be one \textit{k} in each cell}]
    for i in range(self.n):
        for j in range(self.n):
            self += lpSum([self.x[i][j][k]
                            for k in range(self.n)]) == 1, f"cell {i + 1:02d} {j + 1:02d} must be assigned exactly one value"
\end{lstlisting}

\pagebreak

\begin{lstlisting}[caption={Declaring that some cells have already received some initial values}]
    for i in range(self.n):
        for j in range(self.n):
            if self.matrix[i][j]:
                self += self.x[i][j][self.matrix[i][j] -
                                        1] == 1, f"cell {i + 1:02d} {j + 1:02d} has an initial value of {self.matrix[i][j]:02d}"
\end{lstlisting}

\begin{lstlisting}[caption={Performing some preprocessing and invalidating some of the candidate values of each cell}]
    for i in range(self.n):
        for j in range(self.n):
            if not self.matrix[i][j]:
                for value in self.illegal_values(i, j):
                    self += (
                        self.x[i][j][value - 1] == 0,
                        f"cell {i + 1:02d} {j + 1:02d} cannot be assigned a value of {value:02d}"
                    )
\end{lstlisting}

\begin{lstlisting}[caption={Calling the LpProblem solve method and reflecting the result on the two-dimensional array "matrix"}]
    def solve(self, solver=None, **kwargs):

        super().solve(solver=solver, **kwargs)

        if LpStatus[self.status] != "Optimal":
            raise ValueError(
                f"Solver failed with status '{LpStatus[self.status]}'")

        for i in range(self.n):
            for j in range(self.n):
                if not self.matrix[i][j]:
                    self.matrix[i][j] = [
                        self.x[i][j][k].varValue for k in range(self.n)
                    ].index(1) + 1
\end{lstlisting}

\pagebreak

\begin{lstlisting}[caption={Checking for invalid candidate values before even calling LpProblem.solve}]
    def illegal_values(self, row, col):

        values = set()

        for j in range(self.n):
            if self.matrix[row][j] is not None:
                values.add(self.matrix[row][j])

        for i in range(self.n):
            if self.matrix[i][col] is not None:
                values.add(self.matrix[i][col])

        p, q = row // self.m, col // self.m

        for i in range(self.m * p, self.m * (p + 1)):
            for j in range(self.m * q, self.m * (q + 1)):
                if self.matrix[i][j] is not None:
                    values.add(self.matrix[i][j])

        return values
\end{lstlisting}

\pagebreak

If we now print an instance of the previously defined class
we get the following output. \\

\begin{lstlisting}[caption={The objective function and constraints of SudokuLP}]
    sudokulp_solver_9_x_9:

    MINIMIZE
    0*__dummy + 0

    SUBJECT TO
    in_column_01_only_one_01: x_01_01_01 + x_02_01_01 + x_03_01_01 + x_04_01_01
    + x_05_01_01 + x_06_01_01 + x_07_01_01 + x_08_01_01 + x_09_01_01 = 1
    ...
    in_column_09_only_one_09: x_01_09_09 + x_02_09_09 + x_03_09_09 + x_04_09_09
    + x_05_09_09 + x_06_09_09 + x_07_09_09 + x_08_09_09 + x_09_09_09 = 1

    in_row_01_only_one_01: x_01_01_01 + x_01_02_01 + x_01_03_01 + x_01_04_01
    + x_01_05_01 + x_01_06_01 + x_01_07_01 + x_01_08_01 + x_01_09_01 = 1
    ...
    in_row_09_only_one_09: x_09_01_09 + x_09_02_09 + x_09_03_09 + x_09_04_09
    + x_09_05_09 + x_09_06_09 + x_09_07_09 + x_09_08_09 + x_09_09_09 = 1

    in_submatrix_01_01_only_one_01: x_01_01_01 + x_01_02_01 + x_01_03_01
    + x_02_01_01 + x_02_02_01 + x_02_03_01 + x_03_01_01 + x_03_02_01 + x_03_03_01
    = 1
    ...
    in_submatrix_03_03_only_one_09: x_07_07_09 + x_07_08_09 + x_07_09_09
    + x_08_07_09 + x_08_08_09 + x_08_09_09 + x_09_07_09 + x_09_08_09 + x_09_09_09
    = 1

    cell_01_01_must_be_assigned_exactly_one_value: x_01_01_01 + x_01_01_02
    + x_01_01_03 + x_01_01_04 + x_01_01_05 + x_01_01_06 + x_01_01_07 + x_01_01_08
    + x_01_01_09 = 1
    ...
    cell_09_09_must_be_assigned_exactly_one_value: x_09_09_01 + x_09_09_02
    + x_09_09_03 + x_09_09_04 + x_09_09_05 + x_09_09_06 + x_09_09_07 + x_09_09_08
    + x_09_09_09 = 1

    cell_01_08_has_an_initial_value_of_02: x_01_08_02 = 1
    ...
    cell_09_02_has_an_initial_value_of_01: x_09_02_01 = 1

    cell_01_01_cannot_be_assigned_a_value_of_02: x_01_01_02 = 0
    ...
    cell_09_09_cannot_be_assigned_a_value_of_09: x_09_09_09 = 0
\end{lstlisting}

\begin{lstlisting}[caption={The variables of SudokuLP}]
    VARIABLES
    __dummy = 0 Continuous
    0 <= x_01_01_01 <= 1 Integer
    ...
    0 <= x_09_09_09 <= 1 Integer
\end{lstlisting}

\pagebreak

\subsection{The Sudoku X Solver class}

\begin{lstlisting}[caption={Initializing the \textbf{SudokuLP}} super class]
    super().__init__(matrix)
\end{lstlisting}

\begin{lstlisting}[caption={Declaring that there should only be one \textit{k} in the positive diagonal}]
    for k in range(self.n):

        self += lpSum([
            self.x[r][r][k] for r in range(self.n)
        ]) == 1, f"in the diagonal only one {k + 1}"
\end{lstlisting}

\begin{lstlisting}[caption={Declaring that there should only be one \textit{k} in the negative diagonal}]
    for k in range(self.n):

        self += lpSum([
            self.x[r][self.n - 1 - r][k] for r in range(self.n)
        ]) == 1, f"in the anti diagonal only one {k + 1}"
\end{lstlisting}

\pagebreak

\subsection{The Four Square Sudoku Solver class}

\begin{lstlisting}[caption={Initializing the \textbf{SudokuLP}} super class]
    super().__init__(matrix)
\end{lstlisting}

\begin{lstlisting}[caption={Declaring that there should only be one \textit{k} in each shaded square}]
    for i in [1, self.n - self.m - 1]:
        for j in [1, self.n - self.m - 1]:
            for k in range(self.n):
                self += lpSum([
                    [
                        lpSum([
                            self.x[r][c][k]
                            for c in range(j, j + self.m)
                        ])
                    ]
                    for r in range(i, i + self.m)
                ]) == 1, f"in square {i + 1:02d} {j + 1:02d} {i + self.m:02d} {j + self.m:02d} only one {k + 1:02d}"
\end{lstlisting}

\pagebreak

\subsection{The Four Pyramid Sudoku Solver class}

\begin{lstlisting}[caption={Initializing the \textbf{SudokuLP}} super class]
    super().__init__(matrix)
\end{lstlisting}

\begin{lstlisting}[caption={Declaring that there should only be one \textit{k} in the first pyramid-shaped pyramid region}]
    for k in range(1, self.n + 1):
        self += lpSum([
            lpSum([
                self.x[r - 1][c - 1][k - 1]
                for c in range(self.m + r, self.n - r + 1)
            ]) for r in range(1, self.m + 1)
        ]) == 1, f"in pyramid 1 only one {k + 1}"
\end{lstlisting}

\begin{lstlisting}[caption={Declaring that there should only be one \textit{k} in the second pyramid-shaped pyramid region}]
    for k in range(1, self.n + 1):
        self += lpSum([
            lpSum([
                self.x[r - 1][c - 1][k - 1]
                for r in range(1 + c, self.n - self.m + 1 - c + 1)
            ]) for c in range(1, self.m + 1)
        ]) == 1, f"in pyramid 2 only one {k + 1}"
\end{lstlisting}

\pagebreak

\begin{lstlisting}[caption={Declaring that there should only be one \textit{k} in the third pyramid-shaped pyramid region}]
    for k in range(1, self.n + 1):
        self += lpSum([
            lpSum([
                self.x[r - 1][c - 1][k - 1]
                for c in range(self.n + self.m - 1 - r, r - self.m + 1)
            ]) for r in range(self.n - self.m + 1, self.n + 1)
        ]) == 1, f"in pyramid 3 only one {k + 1}"
\end{lstlisting}

\begin{lstlisting}[caption={Declaring that there should only be one \textit{k} in the fourth pyramid-shaped pyramid region}]
    for k in range(1, self.n + 1):
        self += lpSum([
            lpSum([
                self.x[r - 1][c - 1][k - 1]
                for r in range(self.n + self.m + 1 - c, c - 1 + 1)
            ]) for c in range(self.n - self.m + 1, self.n + 1)
        ]) == 1, f"in pyramid 4 only one {k + 1}"
\end{lstlisting}

\begin{lstlisting}[caption={Example Usage}]
    options = {
        "sdk": SudokuLP,
        "sdkx": SudokuXLP,
        "sdkfs": FourSquareSudokuLP,
        "sdkfp": FourPyramidSudokuLP
    }

    extension = path.splitext(args.load)[1][1:]

    matrix = load(args.load)
    problem = options[extension](matrix)
\end{lstlisting}

\pagebreak

\section{Generating Sudoku Puzzles}

\begin{lstlisting}[caption={Generating a new Sudoku Puzzle by transposing the supplied one}]
    def transpose(matrix):
        return list(map(list, [*zip(*matrix)]))
\end{lstlisting}

\begin{lstlisting}[caption={Generating a new Sudoku Puzzle by relabeling the values in the supplied one}]
    def relabel(matrix):
        replacements = {
            original: replacement
            for original, replacement in zip(
                range(1, len(matrix) + 1),
                np.random.permutation(range(1, len(matrix) + 1))
            )
        }

        for i in range(len(matrix)):
            for j in range(len(matrix)):
                if matrix[i][j] is not None:
                    matrix[i][j] = replacements[matrix[i][j]]

        return matrix
\end{lstlisting}

\pagebreak

\begin{lstlisting}[caption={Generating a new Sudoku Puzzle by block reordering the rows and columns of the supplied one}]
    def reorder(matrix):
        def swap_rows(matrix):

            m = int(sqrt(len(matrix)))

            for p in range(m):
                for q in range(m):
                    for r in range(m):
                        i1 = randint(m * p, m * (p + 1) - 1)
                        i2 = randint(m * p, m * (p + 1) - 1)

                        matrix[i1], matrix[i2] = matrix[i2], matrix[i1]

            return matrix

        return transpose(swap_rows(transpose(swap_rows(matrix))))
\end{lstlisting}

\begin{lstlisting}[caption={Example Usage}]
    methods = {
        "transpose": transpose,
        "relabel": relabel,
        "reorder": reorder
    }

    matrix = load(args.load)
    matrix = methods[args.method](matrix)

    dump(matrix, args.save)
\end{lstlisting}

\pagebreak

\section{The Sudoku (.sdk*) file format}

For anyone interested in the details of our implementation,
we now present how sudoku puzzles are being represented internally
as well as how one could load or save a two-dimensional array in .sdk* format. \\

\begin{lstlisting}[caption={A .sdk* file example}]
    9           # +---+---+---+---+---+---+---+---+---+
    1, 1, 5     # | 5 | 3 |   |   | 7 |   |   |   |   |
    1, 2, 3     # +---+---+---+---+---+---+---+---+---+
    1, 5, 7     # | 6 |   |   | 1 | 9 | 5 |   |   |   |
    2, 1, 6     # +---+---+---+---+---+---+---+---+---+
    2, 4, 1     # |   | 9 | 8 |   |   |   |   | 6 |   |
    2, 5, 9     # +---+---+---+---+---+---+---+---+---+
    2, 6, 5     # | 8 |   |   |   | 6 |   |   |   | 3 |
    3, 2, 9     # +---+---+---+---+---+---+---+---+---+
    3, 3, 8     # | 4 |   |   | 8 |   | 3 |   |   | 1 |
    3, 8, 6     # +---+---+---+---+---+---+---+---+---+
    4, 1, 8     # | 7 |   |   |   | 2 |   |   |   | 6 |
    4, 5, 6     # +---+---+---+---+---+---+---+---+---+
    4, 9, 3     # |   | 6 |   |   |   |   | 2 | 8 |   |
    5, 1, 4     # +---+---+---+---+---+---+---+---+---+
    5, 4, 8     # |   |   |   | 4 | 1 | 9 |   |   | 5 |
    5, 6, 3     # +---+---+---+---+---+---+---+---+---+
    5, 9, 1     # |   |   |   |   | 8 |   |   | 7 | 9 |
    6, 1, 7     # +---+---+---+---+---+---+---+---+---+
    6, 5, 2
    6, 9, 6
    7, 2, 6
    7, 7, 2
    7, 8, 8
    8, 4, 4
    8, 5, 1
    8, 6, 9
    8, 9, 5
    9, 5, 8
    9, 8, 7
    9, 9, 9
\end{lstlisting}

\pagebreak

\subsection{Loading Sudoku Puzzles}

\begin{lstlisting}[caption={Removing any useless characters}]
    lines = file.readlines()
    lines = map(lambda line: sub(r"#.*", "", line), lines)
    lines = map(lambda line: sub(r"\s+", "", line), lines)
    lines = enumerate(lines)
    lines = filter(lambda data: len(data[1]) > 0, lines)
\end{lstlisting}

\begin{lstlisting}[caption={Determining the size of the matrix}]
    try:
        index, line = next(lines)

        size = int(line)

        if size <= 0:
            raise ValueError

    except ValueError:
        raise ParseError(
            index, line, "is not a valid size specifier")
\end{lstlisting}

\begin{lstlisting}[caption={Performing some sanity checks corresponding to the size of the matrix}]
    _sqrt = sqrt(size)

    if _sqrt != int(_sqrt):
        raise ParseError(
            index, line, f"{size} is not a perfect square")
\end{lstlisting}

\pagebreak

\begin{lstlisting}[caption={Creating and populating a two-dimensional matrix while performing some sanity checks corresponding to each matrix entry}]
    matrix = [[None for _ in range(size)] for _ in range(size)]

    for index, line in lines:
        try:
            x, y, z = tuple(map(int, line.split(',')))

            if x < 0 or y < 0 or z < 0 or z > size:
                raise IndexError

            if matrix[x - 1][y - 1] is not None:
                raise ParseError(
                    index, line,
                    f"the cell has already been assigned")

            matrix[x - 1][y - 1] = z

        except IndexError:
            raise ParseError(
                index, line,
                f"is not a valid entry for a puzzle of size {size}")

        except ParseError as parse_error:
            raise parse_error

        except:
            raise ParseError(
                index, line, "Malformed entry")

    return matrix
\end{lstlisting}

\pagebreak

\subsection{Dumping Sudoku Puzzles}

\begin{lstlisting}[caption={Dumping a two-dimensional matrix corresponding to a Sudoku Puzzle to a file}]
    file.write(f"{len(matrix)}\n")

    for i in range(len(matrix)):
        for j in range(len(matrix)):
            if matrix[i][j] is not None:
                file.write(f"{i + 1}, {j + 1}, {matrix[i][j]}\n")
\end{lstlisting}

\end{document}
