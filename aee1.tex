\documentclass[12pt]{article}

\usepackage[utf8]{inputenc}
\usepackage[greek,english]{babel}
\usepackage{alphabeta}
\usepackage{latexsym,amsfonts,amssymb,amsthm,amsmath}

\newcommand{\R}{\mathbb{R}}
\newcommand{\N}{\mathbb{N}}
\newcommand{\norm}[1]{\left\lVert#1\right\rVert}
\newcommand{\margin}{\hspace{5pt}}


\newenvironment{rcases}
    {\left.\begin{aligned}}
    {\end{aligned}\right\rbrace}

\setlength{\parindent}{0in}
\setlength{\oddsidemargin}{0in}
\setlength{\textwidth}{6.5in}
\setlength{\textheight}{10in}
\setlength{\topmargin}{-1.5in}
\setlength{\headheight}{18pt}



\title{Αλγοριθμική Επιχειρησιακή Έρευνα - Πρώτη Άσκηση}
\author{Βασίλης Σιώρος \\ Χριστίνα Ανδρινοπούλου}
\date{Οκτώβριος 2019}

\begin{document}

\maketitle 

\vspace{0.1in}

\subsection*{1. Let $C \subseteq \mathbb{R}^n$ be a convex set with $x_1 ,..., x_k \in C$ and let $θ_1 , . . . , θ_k \in \mathbb{R}$ satisfy $θ_i \geq 0$ and
$θ_1 +...+ θ_k = 1$. Show that $θ_1x_1 +...+ θ_kx_k \in C$.}


Θα αποδείξουμε την παραπάνω πρόταση με επαγωγή στο $k$. Για $k = 2$ ισχύει από ορισμό. Κάθε convex set περιέχει το ευθύγραμμο τμήμα που κατασκευάζεται από δύο οποιαδήποτε σημεία του.
\\
\\
ΒΑΣΗ ΕΠΑΓΩΓΗΣ: για $k = 3$: $θ_1x_1 + θ_2x_2 + θ_3x_3 = x$ με $θ_1,θ_2,θ_3 \geq 0$ και $θ_1 + θ_2 + θ_3 = 1$. Σίγουρα ένα εκ των $θ_1,θ_2,θ_3$ θα είναι διάφορο του 1. Χβτγ έστω $θ_1 \neq 1$, άρα: \\
$x = θ_1x_1 +  θ_2x_2 + θ_3x_3 = θ_1x_1 + (1 - θ_1)λ_2x_2 + (1 - θ_1)λ_3x_3$ \\
με $λ_2 = \frac{θ_2}{(1 - θ_1)}$ και $λ_3 = \frac{θ_3}{(1 - θ_1)}$
\\
\\
Αν το ευθύγραμμο τμήμα που σχηματίζεται από τα $x_2$ και $x_3$ ανήκει στο C, τότε και το: \\
$θ_1x_1 + (1 - θ_1)(λ_2x_2 + λ_3x_3)$ θα ανήκει στο C. Πρέπει νδο $λ_2x_2 + λ_3x_3 = y$ ανήκει στο C. \\
$λ_2 +λ_3 = \frac{θ_2}{(1 - θ_1)} + \frac{θ_3}{(1 - θ_1)} = \frac{θ_2 + θ_3}{(1 - θ_1)} = \frac{(1 - θ_1)}{(1 - θ_1)} = 1$ \\
Συνεπώς, αφού $x_2,x_3 \in C$, το y ανήκει στο C, άρα και το x ανήκει στο C, αφού $x_1 \in C$.
\\
\\
ΕΠΑΓΩΓΙΚΗ ΥΠΟΘΕΣΗ: Υποθέτω ότι $θ_1x_1 + ... + θ_{n-1}x_{n-1}$ ανήκει στο C με $θ_i \geq 0$ για $1 \leq i \leq n-1$ και $ \sum_{i=1}^{n-1}θ_i = 1 $ ανήκει στο C.
\\
\\
ΕΠΑΓΩΓΙΚΟ ΒΗΜΑ: Θδο $θ_1x_1 + ... + θ_{n-1}x_{n-1} + θ_nx_n = z$ ανήκει στο C. \\
Χβτγ έστω ότι $θ_n \neq 1$:
$z = θ_nx_n + (1 - θ_n)(λ_1x_1 + ... + λ_{n-1}x_{n-1})$ \\
με $λ_i = \frac{θ_i}{(1 - θ_n)}$ για $1 \leq i \leq n-1$. \\
Όμως, $λ_1x_1 + ... + λ_{n-1}x_{n-1}$ από επαγωγική υπόθεση ανήκει στο C. Άρα, και το z ανήκει στο C.

\vspace{2in}

\subsection*{2. Show that a set is convex if and only if its intersection with any line is convex.}

$\bullet$ Αν έχουμε ένα convex set, τότε η τομή του convex set με μία τυχαιά ευθεία είναι convex. 
\\
\\
Αρχικά, να επισημανθεί ότι μία ευθεία είναι convex set, διότι αν πάρουμε δύο οποιαδήποτε σημεία που ανήκουν στην ευθεία, τότε ανήκει και το ευθύγραμμο τμήμα που κατασκευάζεται από αυτά. Συνεπώς, η ευθεία είναι convex set. \\
Θα αποδείξουμε ότι η τομή δύο convex set είναι convex: \\
Έστω Α και Β είναι δύο convex set. $\forall p_i,p_j \in (A\cap B)$ το ευθύγραμμο τμήμα $p_ip_j$ ανήκει εξ ολοκλήρου στο Α, επειδή Α είναι convex set και ανήκει εξ ολοκλήρου στο Β, επειδή Β είναι convex set, άρα ανήκει εξ ολοκλήρου και στο $Α\cap Β$. Συνεπώς, το $Α\capΒ$ είναι convex set. 
\\
\\
Συνεπώς, η τομή του convex set και μιας ευθείας είναι covex.
\\
\\
$\bullet$ Aν η τομή μιας ευθέιας και ενός set είναι convex, τότε το set είναι convex set.
\\
\\
$\forall x_1,x_2$ που ανήκουν στο set, η τομή του set με την ευθεία που διέρχεται από τα $x_1$ και $x_2$
είναι convex (από υπόθεση), άρα $\forall$ θ με $0 \leq θ \leq 1$ το $θx_1 + (1-θ)x_2$ ανήκουν στην τομή, άρα και στο set. \\
Συνεπώς, το set είναι convex.


\vspace{2in} %Leave more space for comments! \log(\frac{1}{\delta})

\subsection*{3. Show that a set is affine if and only if its intersection with any line is affine.}

$\bullet$ Έστω ένα affine σύνολο S. Θα δείξουμε ότι η τομή του με οποιαδήποτε ευθεία είναι affine σύνολο.\\

Γνωρίζουμε ότι κάθε ευθεία είναι affine σύνολο.
Έστω $S^{'}$ το affine σύνολο που αντιστοιχεί σε κάποια τυχαία ευθεία.
Τότε έχουμε:\\

$x_1 \in S \cap S^{'} \implies x_1 \in S \wedge x_1 \in S^{'}$\\
$x_2 \in S \cap S^{'} \implies x_2 \in S \wedge x_2 \in S^{'}$\\

Δεδομένου ότι τα S και $S^{'}$ είναι affine σύνολα έχουμε:\\

\begin{align*}
    &\begin{rcases}
        x_1 \in S \wedge x_2 \in S & \implies \theta_1 \times x_1 + \theta_2 \times x_2 \in S, \margin \forall \theta_1, \theta_2 \in \R\\
        x_1 \in S^{'} \wedge x_2 \in S^{'} & \implies \theta_1 \times x_1 + \theta_2 \times x_2 \in S^{'}, \margin \forall \theta_1, \theta_2 \in \R
    \end{rcases}
    \Rightarrow
\end{align*}

$x_1 \in S \cap S^{'} \wedge x_2 \in S \cap S^{'} \implies \theta_1 \times x_1 + \theta_2 \times x_2 \in S \cap S^{'}, \margin \forall \theta_1, \theta_2 \in \R$

Επομένως, η τομή του affine συνόλου S με οποιαδήποτε ευθεία είναι affine σύνολο.\\

$\bullet$ Έστω σύνολο S, τέτοιο ώστε η τομή του με οποιαδήποτε ευθεία να είναι affine σύνολο.
Θα δείξουμε ότι το σύνολο S είναι affine.\\

Όπως αναφέρθηκε και προηγουμένως, κάθε ευθεία είναι affine σύνολο.
Έστω, το affine σύνολο $S^{'}$, που αντιστοιχεί στην ευθεία 
που ορίζεται από δύο τυχαία σημεία $x_1, x_2 \in S$.\\

Αφού από υπόθεση το σύνολο $S \cap S^{'}$ είναι affine, έχουμε:\\

$x_1, x_2 \in S \cap S^{'} \implies  \theta_1 \times x_1 + \theta_2 \times x_2 \in S \cap S^{'}, \margin \forall \theta_1, \theta_2 \in \R$\\

και ως εκ τούτου:\\

$x_1, x_2 \in S \implies  \theta_1 \times x_1 + \theta_2 \times x_2 \in S, \margin \forall \theta_1, \theta_2 \in \R$\\

Επομένως, το σύνολο S είναι affine.

\vspace{2in} %Leave more space for comments! \log(\frac{1}{\delta})

\subsection*{4. A set C is midpoint convex, if whenever two points $a, b \in C$, the average or midpoint
$(a + b) / 2$ is in C. Obviously, a convex set is midpoint convex. Prove that if C is closed and
midpoint convex, then C is convex.}

Έστω σημεία $a, b \in C$ και m ένα οποιαδήποτε σημείο επί του ευθύγραμμου τμήματος που ορίζουν τα σημεία a και b.
Ορίζουμε την ακολουθία:\\

\begin{align*}
    m_i & = \frac{a_i + b_i}{2} \\
    \intertext{όπου}
    a_i & = \left\{\begin{array}{lr}
        a, & \text{i = 0}\\
        a_{i - 1}, & i \geq 1 \wedge m_{i - 1} > m\\
        m_{i - 1}, & i \geq 1 \wedge m_{i - 1} \leq m
        \end{array}\right\} \\
    b_i & = \left\{\begin{array}{lr}
        b, & \text{i = 0}\\
        m_{i - 1}, & i \geq 1 \wedge m_{i - 1} > m\\
        b_{i - 1}, & i \geq 1 \wedge m_{i - 1} \leq m
        \end{array}\right\}
\end{align*}\\

Παρατηρούμε τα εξής:\\

$\bullet$ $a_i, b_i, m_i \in C \margin \forall i$\\

Αυτό αποδεικνύεται με μαθηματική επαγωγή στο i ως εξής:\\

ΒΑΣΗ ΕΠΑΓΩΓΗΣ: H πρόταση προς απόδειξη ισχύει για i = 0, καθώς\\

\begin{align*}
    a_0 & = a \in C && \text{ από υπόθεση} \\
    b_0 & = b \in C && \text{ από υπόθεση} \\
    m_0 & = \frac{a_0 + b_0}{2} = \frac{a + b}{2} \in C && \text{ λόγω midpoint covexivity}
\end{align*} \\

ΕΠΑΓΩΓΙΚΗ ΥΠΟΘΕΣΗ: Έστω ότι $a_i, b_i, m_i \in C \margin \forall i \leq n$ \\

ΕΠΑΓΩΓΙΚΟ ΒΗΜΑ: Θα δείξουμε ότι $a_{n + 1}, b_{n + 1}, m_{n + 1} \in C$ \\

Υποθέτουμε χωρίς βλάβη της γενικότητας ότι $m_n > m$. Έτσι έχουμε: \\

\begin{align*}
    a_{n + 1} & = a_n \in C && \text{ από επαγωγική υπόθεση}\\
    b_{n + 1} & = m_n \in C && \text{ από επαγωγική υπόθεση} \\
    m_{n + 1} & = \frac{a_{n + 1} + b_{n + 1}}{2} = \frac{a_n + m_n}{2} \in C && \text{ λόγω midpoint covexivity}
\end{align*}

$\bullet$ Το σημείο m βρίσκεται πάντα επί του ευθύγραμμου τμήματος $\overleftrightarrow{a_{i}b_{i}}$,
γεγονός προφανές από τον ορισμό της ακολουθίας $m_i$.
Επομένως, ισχύει η ανισότητα $\norm{m_i - m} \leq \norm{a_i - b_i}$. \\

Θα αποδείξουμε με μαθηματική επαγωγή στο i, ότι $\norm{a_i - b_i} = \norm{a - b} \times 2^{-i}$\\

ΒΑΣΗ ΕΠΑΓΩΓΗΣ: Για i = 0, έχουμε: $\norm{a_0 - b_0} = \norm{a - b} = \norm{a - b} \times 2^{-0}$\\

ΕΠΑΓΩΓΙΚΗ ΥΠΟΘΕΣΗ: Έστω ότι $\norm{a_i - b_i} = \norm{a - b} \times 2^{-i} \margin \forall i \leq n$\\

ΕΠΑΓΩΓΙΚΟ ΒΗΜΑ: Υποθέτοντας χωρίς βλάβη της γενικότητας ότι $m_n > m$,
όπως και παραπάνω, για i = n + 1, έχουμε:\\

\begin{align*}
    \norm{a_{n + 1} - b_{n + 1}} & = \norm{a_n - m_n} && \Leftrightarrow \\
    \norm{a_{n + 1} - b_{n + 1}} & = \norm{a_n - \frac{a_n + b_n}{2}} && \Leftrightarrow \\
    \norm{a_{n + 1} - b_{n + 1}} & = \norm{\frac{b_n - a_n}{2}} && \Leftrightarrow \\
    \norm{a_{n + 1} - b_{n + 1}} & = \norm{\frac{a_n - b_n}{2}} && \Leftrightarrow \\
    \norm{a_{n + 1} - b_{n + 1}} & = \norm{a_n - b_n} \times {2^{-1}} && \Leftrightarrow \\
    \norm{a_{n + 1} - b_{n + 1}} & = \norm{a - b} \times 2^{-n} \times {2^{-1}} && \Leftrightarrow \\
    \norm{a_{n + 1} - b_{n + 1}} & = \norm{a - b} \times 2^{-(n + 1)} && \Leftrightarrow
\end{align*}\\

Έτσι, έχουμε: $0 \leq \norm{m_i - m} \leq \norm{x - y} \times 2^{-i}$\\

Είναι προφανές πως η ακολουθία $\norm{m_i - m}$ είναι φθίνουσα και λόγω του ότι είναι
και φραγμένη συγκλίνει στο μέγιστο κάτω φράγμα της, δηλαδή:\\

$\lim_{i \to \infty} \norm{m_i - m} = 0 \Leftrightarrow \lim_{i \to \infty} m_i = m$\\

Τέλος, επειδή το σύνολο είναι κλειστό περιέχει τα όρια ακολουθιών στοιχείων του και αφού,
όπως δείξαμε $m_i \in C \margin \forall i$, συνεπάγεται ότι $m \in C$. Αυτό ισχύει για
οποιαδήποτε σημείο m επί του ευθύγραμμου τμήματος $\overleftrightarrow{αβ}$ και άρα
$a, b \in C \implies m = \theta_1 \times a + \theta_2 \times b \in C \margin \forall \theta_1, \theta_2 \in \R$

\vspace{2in} %Leave more space for comments! \log(\frac{1}{\delta})

\subsection*{5. Show that the convex hull of a set S is the intersection of all convex sets that contain
S.(The same method can be used to show that the conic, or affine, or linear hull of a set S is
the intersection of all conic sets, or affine sets, or subspaces that contain S.}

Το convex hull ενός συνόλου S, ορίζεται ως εξής:\\

$conv(S) = \lbrace \sum_{i = 1}^{n} λ_i \times s_i
\mid
s_i \in S, \margin
n \in \N, \margin
λ_i \geq 0 \margin \forall i, \margin
\sum_{i = 1}^{n} λ_i = 1
\rbrace$\\

$\bullet$ Αρχικά θα αποδείξουμε ότι η τομή όλων των convex υπερσυνόλων του S,
είναι υπερσύνολο του convex hull του.

Έστω convex σύνολο C, τέτοιο ώστε $C \supseteq S$.\\

Δεδομένου ότι το σύνολο C είναι convex, ισχύει:\\

$\sum_{i = 1}^{n} λ_i * c_i \in C \text{ όπου }
c_i \in S, \margin
n \in \N, \margin
λ_i \geq 0 \margin \forall i, \margin
\sum_{i = 1}^{n} λ_i = 1$\\

Δεδομένου τώρα ότι $C \supseteq P$, ισχύει:\\

$s_i \in S \Rightarrow s_i \in C \Rightarrow \sum_{i = 1}^{n} λ_i \times s_i \in C$\\

Ως εκ τούτου, προκύπτει $C \supseteq conv(S)$ και αφού αυτό ισχύει για οποιαδήποτε convex
υπερσύνολο του S συνεπάγεται πως και η τομή αυτών των συνόλων είναι υπερσύνολο του conv(S).\\

$\bullet$ Στη συνέχεια θα δείξουμε ότι το convex hull του συνόλου S είναι υπερσύνολο της
τομής όλων των convex υπερσυνόλων του.\\

Είναι προφανές πως $conv(S) \supseteq S$. Αρκεί να δείξουμε ότι το σύνολο conv(S)
είναι convex, καθώς σε αυτή την περίπτωση θα ανήκει στο σύνολο convex υπερσυνόλων του
S και ως εκ τούτου θα είναι υπερσύνολο και της τομής τους.\\

Έστω δύο σημεία $x_1$ και $x_2$, τέτοια ώστε:\\

\begin{align*}
    x_1 & = \sum_{i = 1}^{n} λ_i \times s_i && \in conv(S) \\
    x_2 & = \sum_{i = 1}^{n} μ_i \times s_i && \in conv(S)
\end{align*}\\

Έστω τώρα $\theta \in \lbrack 0, 1 \rbrack$\\

\begin{align*}
    \theta \times x_1 + (1 - \theta) \times x_2 & = \theta \times \sum_{i = 1}^{n} λ_i \times s_i + (1 - \theta) \times \sum_{i = 1}^{n} μ_i \times s_i && \Leftrightarrow \\
    \theta \times x_1 + (1 - \theta) \times x_2 & = \sum_{i = 1}^{n}(\theta \times λ_i + (1 - \theta) \times μ_i) \times s_i
\end{align*}\\

Δεδομένου ότι:\\

\begin{align*}
    \sum_{i = 1}^{n}(\theta \times λ_i + (1 - \theta) \times μ_i) & = \theta \times \sum_{i = 1}^{n} λ_i + (1 - \theta) \times \sum_{i = 1}{n} μ_i && \Leftrightarrow \\
    \sum_{i = 1}^{n}(\theta \times λ_i + (1 - \theta) \times μ_i) & = \theta \times 1 + (1 - \theta) \times 1 && \Leftrightarrow \\
    \sum_{i = 1}^{n}(\theta \times λ_i + (1 - \theta) \times μ_i) & = 1
\end{align*}\\

και\\

$\theta \times λ_i + (1 - \theta) \times μ_i \geq 0 \margin \forall i \in \lbrace 1 \dotsc n \rbrace$\\

καταλήγουμε στο συμπέρασμα $\theta \times x_1 + (1 - \theta) \times x_2 \in conv(S)$
και άρα το σύνολο conv(S) είναι convex και ως εκτούτου είναι και υπερσύνολο της
τομής των convex υπερσυνόλων του S.\\

Από τα παραπάνω προκύπτει\\

\begin{align*}
    &\begin{rcases}
        conv(S) & \supseteq \bigcap C \text{ όπου C οποιοδήποτε convex υπερσύνολο του S} \\
        conv(S) & \subseteqq \bigcap C \text{ όπου C οποιοδήποτε convex υπερσύνολο του S}
    \end{rcases}
    \Rightarrow
\end{align*}

$conv(S) \equiv \bigcap C \text{ όπου C οποιοδήποτε convex υπερσύνολο του S}$

\vspace{2in} %Leave more space for comments! \log(\frac{1}{\delta})

\subsection*{6. What is the distance between two parallel hyperplanes ${x \in \mathbb{R}^n : a^Tx = b_1 }$ and
${x \in \mathbb{R}^n : a^Tx = b_2 }$?}

Τα δύο hyperplanes, έστω $B_1$ και $B_2$ είναι μεταξύ τους παράλληλα, άρα έχει νόημα να μελετήσουμε την απόσταση μεταξύ τους. Αν δεν ήταν παράλληλα, η απόσταση μεταξύ των $B_1$ και $B_2$ θα ήταν 0, αφού θα υπήρχε σημείο τομής ανάμεσα τους. \\
Έστω ένα σημείο $x_1$ που ανήκει στο $B_1$ και έστω μία ευθεία Ε, η οποία διέρχεται από το $x_1$ και είναι κάθετη στο $Β_1$. Η Ε τέμνει το $B_2$ στο σημείο $x_2$ επίσης κάθετα, γιατί τα δύο hyperplanes είναι μεταξύ τους παράλληλα. Αρκεί να βρούμε την απόσταση μεταξύ των δύο σημείων $x_1$ και $x_2$, δηλαδη να βρούμε την $\left\|x_1-x_2\right\|_2 $. \\
Η ευθεία Ε έχει ίδια κατεύθυνση με το α, γιατί και το α και η ευθεία Ε είναι κάθετα στα δύο hyperplanes, συνεπώς την ευθεία μπορούμε να την γράψουμε ως: $αy + x_1$. \\
Η τομή της Ε με το $Β_2$ είναι: $α^{Τ}x = b_2 \Leftrightarrow α^{Τ}(αy + x_1) = b_2 \Leftrightarrow α^{Τ}αy + α^{Τ}x_1 = b_2 \Leftrightarrow \\ \Leftrightarrow y = \frac{b_2 - α^{Τ}x_1}{α^{Τ}α} \Leftrightarrow y = \frac{b_2 - b_2}{α^{Τ}α} $ \\
Άρα, $x_2 = αy + x_1 \Leftrightarrow x_2 = a\frac{b_2 - b_1}{α^{Τ}α} + x_1$. \\
Συνεπώς, $\left\|x_1-x_2\right\|_2 = \left\|x_1 - \frac{b_2 - b_1}{α^{Τ}} - x_1\right\|_2 = \left\|-\frac{b_2 - b_1}{α^{Τ}}\right\|_2 = \frac{|b_2 - b_1|}{\left\|α\right\|_2}$


\vspace{2in} %Leave more space for comments! \log(\frac{1}{\delta})

\subsection*{7. Let a and b be distinct points in $\mathbb{R}^n$ . Show that the set of all points that are closer (in
Euclidean norm) to a than b is a halfspace.}

Έστω $S = \lbrace x \in \R \mid \norm{x - a}_2 \leq \norm{x - b}_2 \rbrace$ το σύνολο
των σημείων $x \in \R$, τα οποία βρίσκονται πιο κοντά στο a, σε σχέση με το b
βάσει Ευκλείδιας Νόρμας.\\

\begin{align*}
    \norm{x - a}_2 & \leq \norm{x - b}_2 && \Leftrightarrow \\    
    \norm{x - a}_2^2 & \leq \norm{x - b}_2^2 && \Leftrightarrow \\    
    (x - a)^T \times (x - a) & \leq (x - b)^T \times (x - b) && \Leftrightarrow \\
    x^T \times x - x^T \times a - a^T \times x + a^T \times a & \leq x^T \times x - x^T \times b - b^T \times x + b^T \times b && \Leftrightarrow \\
    2 \times b^T \times x - 2 \times a^T \times x & \leq b^T \times b - a^T \times a && \Leftrightarrow \\
    2 \times (b - a)^T \times x & \leq b^T \times b - a^T \times a
\end{align*}\\

Θέτουμε\\

\begin{align*}
    c & = 2 \times (b - a) \\
    d & = b^t \times b - a^T \times a
\end{align*}

και λαμβάνουμε την ανισότητα $c^T \times x \leq d$, η οποία ορίζει έναν κλειστό ημιχώρο.

\end{document}
